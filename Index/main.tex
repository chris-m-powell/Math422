\documentclass[10pt]{article}

\usepackage{fancyhdr}
\usepackage{amsmath,amssymb,amsthm}
\usepackage{mathrsfs}
\usepackage{enumitem}
\usepackage{verbatim}
\usepackage{parskip}
\usepackage{xcolor}
\usepackage{array} 
\usepackage{url} 
\usepackage{float}
\usepackage{braket}
\usepackage{listings}
\usepackage{inconsolata}
\usepackage{mystyle}
\usepackage{lastpage}
\usepackage{fontspec}
\usepackage{sectsty}
\usepackage[euler-digits,euler-hat-accent]{eulervm}

\raggedbottom
\setmainfont{URW Palladio L}
% \newfontfamily\headingfont[Path = /usr/share/fonts/opentype/bebas-neue/]{BebasNeue Regular.otf}
% \setmainfont[Path = /usr/share/fonts/opentype/tex-gyre-pagella/]{texgyrepagella-regular.otf}
\begin{document}
\title{\vspace{-2em}Definitions, Theorems, etc.\vspace{-1em}}
\author{}
\date{}
\maketitle
\thispagestyle{fancy}

\sectionfont{\fontsize{12}{15}\selectfont}
\vspace{-4em}
\setcounter{section}{1}
\section{Pythagorean Triples}

\begin{definition*}[\textcolor{red}{Primitive Pythagorean Triple}] A \emph{primitive Pythagorean triple} (PPT) is a triple of numbers $(a,b,c)$ such that $a$, $b$, $c$ have no common factors and satisfy
    \[a^2+b^2=c^2.\]
\end{definition*}

\begin{theorem}[\textcolor{red}{Pythagorean Triple Theorem}] We will get every primitive Pythagorean triple of $(a,b,c)$ with $a$ odd and $b$ even using the formulas
    \[a=st, \quad b=\frac{s^2-t^2}{2}, \quad c=\frac{s^2+t^2}{2},\]
    where $s > t \geq 1$ are chosen to be any odd integers with no common factors.
\end{theorem}

\section{Pythagorean Triples and the Unit Circle}

\begin{theorem} Every point on the circle
    \[x^2+y^2=1\]
    whose coordinates are rational numbers can be obtained from the formula
    \[(x,y)=\left( \frac{1-m^2}{1+m^2},\frac{2m}{1+m^2} \right) \]
    by substituting in rational numbers for $m$ (except for the point $(-1,0)$ which is the limiting value as $m \rightarrow \infty$).
\end{theorem}


\section{Sums of Higher Powers and Fermat's Last Theorem}

\begin{theorem*}[\textcolor{red}{Fermat's Last Theorem}]
No three positive integers $a$, $b$, and $c$ satisfy the equation
    \[a^n+b^n=c^n\]
    for all $n \geq 3$.
\end{theorem*}

\section{Divisibility and the Greatest Common Divisor}

\begin{definition*}[\textcolor{red}{Greatest Common Divisor}] The \emph{greatest common divisor} of two numbers $a$ and $b$ (not both zero) is the largest number that divides them both. It is denoted by $\gcd(a,b)$.
\end{definition*}

\begin{definition*}[\textcolor{red}{Relatively Prime}] If $\gcd(a,b)=1$, then we say that $a$ and $b$ are \emph{relatively prime}.
\end{definition*}

\begin{definition*}[\textcolor{red}{Least Common Multiple}] A number $L$ is called a \emph{common multiple} of $m$ and $n$ if both $m$ and $n$ divide $L$. The smallest such $L$ is called the \emph{least common multiple of $m$ and $n$} and is denoted by $\lcm(m,n)$.
\end{definition*}

\begin{theorem*}Let $m,n \in \Z$. Then
    \[\lcm(m,n) = \frac{mn}{\gcd(m,n)}.\]
\end{theorem*}

\begin{theorem}[\textcolor{red}{Euclidean Algorithm}] To compute the greatest common divisor of two numbers $a$ and $b$, let $r_{-1} = a$, let $r_0=b$, and compute successive quotients and remainders
    \[r_{i-1} = q_{i+1} \cdot r_i + r_{i+1}\]
for $i=0,1,2,\ldots$ until some remainder $r_{n+1}$ is $0$. The last nonzero remainder $r_n$ is the greast common divisor of $a$ and $b$.
\end{theorem}


\section{Linear Equations and the Greatest Common Divisor}

\begin{theorem}[\textcolor{red}{Linear Equation Theorem}] Let $a$ and $b$ be nonzero integers,
    and let $g=\gcd(a,b)$. Then the equation
    \[ax+by=g\]
    always has a solution $(x_1,y_1)$ in integers, and this solution can be found by the Euclidean algorithm. Then every solution to the equation can be obtained by substituting integers $k$ into the formula
    \[\left( x_1 + k \cdot \frac{b}{g}, y_1-k \cdot \frac{a}{g} \right).\]
\end{theorem}

\begin{comment}
\begin{lemma*} Let $a,b,c \in \Z$. If $\gcd(a,b,c)=1$, then $\gcd(\gcd(a,b),c)=1$.
\end{lemma*}

\begin{theorem*} Let $a,b,c \in \Z$. Then the equation
    \[ax+by+cz=1\]
    has solutions $(x,y,z) \in \Z^3$ if, and only if, $\gcd(a,b,c)=1$.
\end{theorem*}
\end{comment}

\section{Factorization and the Fundamental Theorem of Arithmetic}

\begin{definition*}[\textcolor{red}{Prime}] A \emph{prime} integer is an integer $p \geq 2$ whose only (positive) divisors are $1$ and $p$.
\end{definition*}


\begin{definition*}[\textcolor{red}{Composite}] Integers $m \geq 2$ that are not primes are called \emph{composite} numbers.
\end{definition*}

\begin{comment}
\begin{lemma} Let $p$ be a prime number, and suppose that $p$ divides the product $ab$. Then either $p$ divides $a$ or $p$ divides $b$ (or $p$ divides both $a$ and $b$).
\end{lemma}
\end{comment}

\setcounter{theorem}{1}
\begin{theorem}[\textcolor{red}{Prime Divisibility Property}] Let $p$ be a prime number, and suppose that $p$ divides the product $a_1a_2 \cdots a_r$. Then $p$ divides at least one of the factors $a_1,a_2,\ldots,a_r$.
\end{theorem}

\begin{theorem}[\textcolor{red}{The Fundamental Theorem of Arithmetic}] Every integer $n \geq 2$ can be factored in a product of primes
    \[n = p_1p_2 \cdots p_r\]
    in exactly one way (up to rearrangement).
\end{theorem}

\section{Congruences}

\begin{definition*}[\textcolor{red}{Congruence}] An integer $a$ is \emph{congruent} to $b$ modulo $m$, and we write
    \[a \equiv b \mod{m},\]
    if $m$ divides $a-b$.
\end{definition*}


\begin{proposition*} $\equiv$ is an equivalence relation.
\end{proposition*}


\begin{proposition*} Let $a,b,c,d \in \Z$. Assume $a \equiv b \mod{m}$ and $c \equiv d \mod{m}$. Then 
    \begin{enumerate}[label=(\roman*)]
        \item $a+c \equiv b + d \mod{m}$
        \item $ac \equiv bd \mod{m}$.
    \end{enumerate}
\end{proposition*}


\begin{theorem}[\textcolor{red}{Linear Congruence Theorem}] Let $a$, $c$ and $m$ be integers with $m \geq 1$, and let $g= \gcd(a,m)$.
    \begin{enumerate}[label=(\alph*)]
        \item If $g \nmid c$, then the congruence $ax \equiv c \mod{m}$ has no solutions.
        \item If $g \mid c$, then the congruence $ax \equiv c \mod{m}$ has exactly $g$ incongruent solutions. To find the solutions, first find a solution $(u_0,v_0)$ to the linear equation
            \[au+mv=g\]
    Then $x_0=cu_0/g$ is a solution to $ax \equiv c \mod {m}$, and a complete set of incongruent solutions is given by 
            \[x \equiv x_0 + k \cdot \frac{m}{g} \mod{m} \quad \textnormal{for $k = 0,1,\ldots,g-1$}.\]
    \end{enumerate}
\end{theorem}


\begin{theorem}[\textcolor{red}{Polynomial Roots modulo $p$ Theorem}] Let $p$ be a prime number and let 
    \[f(x) = a_0d^d+a_1 d^{d-1} + \cdots + a_d\]
    be a polynomial of degree $d \geq 1$ with integer coefficients and with $p \nmid a_0$. Then the congruence
    \[f(x) \equiv 0 \mod{p}\]
    has at most $d$ incongruent solutions.
\end{theorem}


\section{Congruences, Powers, and Fermat's Little Theorem}

\begin{theorem}[\textcolor{red}{Fermat's Little Theorem}] Let $p$ be a prime number, and let $a$ be any number with $a \not\equiv 0 \mod{p}$. Then
    \[a^{p-1} \equiv 1 \mod{p}.\]
\end{theorem}

\begin{comment}
\setcounter{lemma}{1}
\begin{lemma}Let $p$ be a prime number and let $a$ be a number with $a \neq 0\mod{p}$. Then the numbers
    \[a,2a,3a,\ldots,(p-1)a \mod{p}\]
    are the same as the numbers
    \[1,2,3,\ldots,(p-1) \mod{p},\]
    although they may be in a different order.
\end{lemma}
\end{comment}

\section{Congruences, Powers, and Euler's Formula}

\begin{definition*}[\textcolor{red}{Euler Phi Function}]\emph{Euler's phi function} is the is the function $\varphi(m):\N \to \N$ defined by
    \[\varphi(m)=\# \set{a : 1 \leq a \leq m \ \textnormal{and} \ \gcd(a,m)=1}.\]
\end{definition*}

\begin{theorem}[\textcolor{red}{Euler's Formula}] If $\gcd(a,m)=1$, then
    \[a^{\varphi(m)}=1 \mod{m}.\]
\end{theorem}

\begin{comment}
\setcounter{lemma}{1}
\begin{lemma} If $\gcd(a,m)=1$, then the numbers
    \[b_1a,b_2a,b_3a,\ldots,b_{\varphi(m)}a \mod{m}\]
    are the same as the numbers
    \[b_1,b_2,b_3,\ldots,b_{\varphi(m)} \mod{m}\]
    although they may be in a different order.
\end{lemma}
\end{comment}

\section{Euler's Phi Function and the Chinese Remainder Theorem}


\begin{theorem}[\textcolor{red}{Phi Function Formulas}]
    \begin{enumerate}[label=(\alph*)]
        \item If $p$ is prime and $k \geq 1$, then
            \[\varphi\left(p^k \right)=p^k - p^{k-1}.\]
        \item If $\gcd(m,n)=1$, then $\varphi(mn)=\varphi(m)\varphi(n)$.
    \end{enumerate}
\end{theorem}

\begin{corollary*}
    Let $m$ be a positive integer and suppose $p_1,\ldots,p_r$ are the distinct primes that divide $m$. Then
    \[\varphi(m) = m \prod_{i=1}^{r}\left( 1 - \frac{1}{p_i}\right).\]
\end{corollary*}

\begin{comment}
\begin{theorem}[Chinese Remainder Theorem]
Let $m$ and $n$ be integers satisfying $\gcd(m,n)=1$, and let $b$ and $c$ be any integers. Then the simultaneous congruences
    \[x \equiv b \mod{m} \quad \textnormal{and} \quad x \equiv c \mod{n}\]
have exactly one solution with $0 \leq x < mn$.
\end{theorem}

\begin{theorem*}[Chinese Remainder Theorem]
    (Version 2) Let $m,n$ be integers satisfying $\gcd(m,n)=1$, and let $b$ and $c$ be any integers. Then there is a unique $x \in \Z_{mn}$ such that
    \[x \equiv b \mod{m} \quad \textnormal{and} \quad x \equiv c \mod{n}.\]
\end{theorem*}
\end{comment}

\begin{theorem*}[\textcolor{red}{Generalized Chinese Remainder Theorem}]
Let $m_1,\ldots,m_n \in \Z$ such that $\gcd(m_i,m_j)=1$ for all $1 \leq i,j \leq n$ with $i \neq j$. Let $a_1,\ldots,a_n \in \Z$. Then the system of congruences
    \begin{align*}
        x &\equiv a_1 \mod{m_1}
    \\  x &\equiv a_2 \mod{m_2}
        \\&\vdots
    \\  x &\equiv a_n \mod{m_n}
\end{align*}
    has a unique solution modulo $M=\prod_{i=1}^n m_i$, given by 
    \[x \equiv \sum_{i=1}^n a_i \left(\frac{M}{m_i} \right) y_i,\]
    where $y_i \equiv \left(\frac{M}{m_i}\right)^{-1} \mod{m_i}$ for all $1 \leq i \leq n$.
\end{theorem*}

\section{Prime Numbers}

\begin{theorem}[\textcolor{red}{Infinitude of Primes}]
There are infinitely many prime numbers.
\end{theorem}

\begin{comment}
\begin{theorem}[Infinitude of Primes $3 \mod{4}$]
    There are infinitely many primes that are congruent to $3$ modulo $4$.
\end{theorem}
\end{comment}

\begin{theorem}[\textcolor{red}{Dirichlet's Theorem on Primes in Arithmetic Progression}]
    Let $a$ and $m$ be integers with $\gcd(a,m)=1$. Then there are infinitely primes that are congruent to $a$ modulo $m$. That is, there are infinitely many prime numbers $p$ satisfying
    \[p \equiv a \mod{m}.\]
\end{theorem}

\begin{comment}
\section{Counting Primes}

\begin{theorem}[The Prime Number Theorem]
    When $x$ is large, the number of primes less than $x$ is approximately equal to $x/\ln(x)$. In other words,
    \[\lim_{x \to \infty} \frac{\pi(x)}{x/\ln(x)}=1.\]
\end{theorem}

\setcounter{conjecture}{1}
\begin{conjecture}[Goldbach's Conjecture]
    Every even number $n \geq 4$ is a sum of two primes.
\end{conjecture}

\begin{conjecture}[Twin Primes Conjecture]
There are infinitely many prime numbers $p$ such that $p+2$ is also prime.
\end{conjecture}
\end{comment}

\setcounter{section}{15}
\section{Powers Modulo $m$ and Successive Squaring}

\begin{algorithm}[\textcolor{red}{Successive Squaring to Compute $a^k$ modulo $m$}]
    The following steps compute the value of $a^k \mod{m}$:
    \begin{enumerate}
        \item Write $k$ as a sum of powers of $2$.
            \[k=u_0+u_1\cdot 2 +u_2 \cdot 4 + u_3 \cdot 8 + \cdots + u_r \cdot 2^r,\]
            where each $u_i$ is either $0$ or $1$. (This is called the binary expansion of $k$.)
        \item Make a table of powers of $a$ modulo $m$ using successive squaring.
            \begin{center} 
            {\renewcommand{\arraystretch}{1.0}
                \begin{tabular}{ c c c c }
            \hline
                    \\
                $a^1$ &             &                   & $\equiv A_0 \mod{m}$ \\
                $a^2$ & $\equiv(a^1)^2$   & $\equiv A_0^2$    & $\equiv A_1 \mod{m}$ \\
                $a^4$ & $\equiv(a^2)^2$   & $\equiv A_1^2$    & $\equiv A_2 \mod{m}$ \\
                $a^8$ & $\equiv(a^4)^2$   & $\equiv A_2^2$    & $\equiv A_3 \mod{m}$ \\
                        & $\vdots$        &                   & $\vdots$ \\
                $a^{2r}$ & $\equiv \left(a^{2^{r-1}} \right)^2$   & $\equiv A_{r-1}^2$    & $\equiv A_r \mod{m}$ \\
\\
                \hline
                \end{tabular}}
            \end{center}
        \item The product
            \[A_0^{u_0}\cdot A_1^{u_1} \cdot A_2^{u_2} \cdots A_r^{u_R} \mod{m}\]
            will be congruent to $a^k \mod{m}$. Note that all the $u_i$'s are either $0$ or $1$, so this number is really the product of those $A_i$'s for which $u_i$ equals $1$.
    \end{enumerate}
\end{algorithm}

\section{Computing $k$\textsuperscript{th} Roots Modulo $m$}

\begin{comment}
\begin{theorem*}[\textcolor{red}{$k$\textsuperscript{th} Roots Modulo $m$}]
Let $b$, $k$, and $m$ be integers. Suppose $\gcd(b,m)=1$ and $\gcd(k, \varphi(m))=1$.    Assume $ku \equiv 1 \mod{\varphi(m)}$.
If $x^k \equiv b \mod{m}$, then $x \equiv b^u \mod{m}$.
\end{theorem*}
\end{comment}

\begin{algorithm}[\textcolor{red}{How to Compute $k$\textsuperscript{th} Roots modulo $m$}]
Let $b$, $k$, and $m$ be given integers that satisfy
    \[\gcd(b,m)=1 \quad \textnormal{and} \quad \gcd(k, \varphi(m))=1.\]
    The following steps give a solution to the congruence
    \[x^k \equiv b \mod{m}.\]
    \begin{enumerate}
        \item Compute $\varphi(m)$.
        \item Find positive integers $u$ and $v$ that satisfy $ku-\varphi(m)v=1$.
        \item Compute $b^u \mod{m}$ by successive squaring. The value obtained gives the solution $x$
    \end{enumerate}
\end{algorithm}

\section{Powers, Roots, and "Unbreakable" Codes}

\setcounter{section}{19}
\section{Squares Modulo $p$}

\begin{definition*}[\textcolor{red}{Quadratic Residue modulo $p$}]
    A nonzero number that is congruent to a square modulo a prime $p$ is called a \textit{quadratic residue modulo $p$}; otherwise, it is called a \textit{nonresidue modulo $p$}.
\end{definition*}

\begin{theorem}Let $p$ be an odd prime. Then there are exactly $(p-1)/2$ quadratic residues modulo $p$ and exactly $(p-1)/2$ nonresidues modulo $p$.
\end{theorem}

\begin{comment}
\begin{theorem}[Quadratic Residue Multiplication Rule] 
(Version 1) Let $p$ be an odd prime. Then:
    \begin{enumerate}[label=(\roman*)]
        \item The product of two quadratic residues modulo $p$ is a quadratic residue.
        \item The product of a quadratic residue and a nonresidue is a non residue.
        \item The product of two non residues is a quadratic residue.
    \end{enumerate}
These three rules can be summarized symbolically by the formulas
    \[QR \times QR = QR, \quad QR \times NR = NR, \quad NR \times NR = QR.\]
\end{theorem}
\end{comment}

\begin{definition*}[\textcolor{red}{Legendre Symbol of $a$ modulo $p$}]
    The \emph{Legendre symbol of $a$ modulo $p$} is
    \[\legendre{a}{p} = \begin{array}{cc}  
        \begin{cases}
        1 & \textnormal{if $a$ is a quadratic residue modulo $p$} \\
        -1 & \textnormal{if $a$ is a nonresidue modulo $p$.}
    \end{cases}
\end{array}\]
\end{definition*}

\begin{theorem}[\textcolor{red}{Quadratic Residue Multiplication Rule}]
    Let $p$ be an odd prime. Then
    \[\legendre{a}{p}\legendre{b}{p}=\legendre{ab}{p}.\] 
\end{theorem}

\section{Is $-1$ a Square Modulo $p$}

\begin{theorem}[\textcolor{red}{Euler's Criterion}]
    Let $p$ be an odd prime. Then
    \[a^{(p-1)/2} \equiv \legendre{a}{p} \mod{p}.\]
\end{theorem}

\begin{comment}
\begin{theorem}[Quadratic Reciprocity]
    (Part I) Let $p$ be an odd prime. Then
    \begin{align*}
        -1 \ \textnormal{is a quadratic residue modulo $p$} \  & \textnormal{if $p\equiv 1 \mod{4}$, and} 
        \\ -1 \ \textnormal{is a nonresidue modulo $p$} \ & \textnormal{if $p \equiv 3 \mod{4}$.}
    \end{align*}
\end{theorem}

\begin{theorem}[Primes $1 \mod{4}$ Theorem]
There are infinitely many primes congruent to $1$ modulo $4$.
\end{theorem}

\begin{theorem}[Quadratic Reciprocity]
    (Part II) Let $p$ be an odd prime. Then $2$ is a quadratic residue modulo $p$ if $p$ is congruent to $1$ or $7$ modulo $8$, and $2$ is a nonresidue modulo $p$ if $p$ is congruent to $3$ or $5$ modulo $8$. In terms of the Legendre symbol,
    \[\legendre{2}{p}=\begin{cases}
        1 & \textnormal{if $p \equiv 1$ or $7 \mod{8}$,}
        \\-1 & \textnormal{if $p \equiv 3$ or $5 \mod{8}$}.
    \end{cases}\]
\end{theorem}
\end{comment}

\section{Quadratic Reciprocity}

\begin{theorem}[\textcolor{red}{Law of Quadratic Reciprocity}]
Let $p$ and $q$ be distinct odd primes.
    \begin{align*}
        \legendre{-1}{p}&= 
        \begin{cases}
            1 & \textnormal{if $p\equiv 1 \mod{4}$,} \\
            -1 & \textnormal{if $p \equiv 3 \mod{4}$.}
        \end{cases}
        \\ \legendre{2}{p}&=
        \begin{cases}
            1 & \textnormal{if $p\equiv 1$ or $7 \mod{8}$,} \\
            -1 & \textnormal{if $p \equiv 3$ or $5 \mod{8}$.}
        \end{cases}
        \\ \legendre{q}{p}&=
        \begin{cases}
            \legendre{p}{q} & \textnormal{if $p\equiv 1 \mod{4}$ or $q \equiv 1 \mod{4}$,} \\
            -\legendre{p}{q} & \textnormal{if $p \equiv 3 \mod{4}$ and $q \equiv 3\mod{4}$.}
        \end{cases}
    \end{align*}
\end{theorem}


\begin{definition*}[\textcolor{red}{Jacobi Symbol of $a$ modulo $b$}]
    Let $a$ and $b$ be odd positive integers. Suppose $b$ has factorization $b=\prod_{i=1}^r p_r$, where each $p_i$ is a distinct prime. Then the \textit{Jacobi symbol of $a$ modulo $b$}is
    \[\legendre{a}{b}=\prod_{i=1}^r \legendre{a}{p_i}.\]
\end{definition*}


\begin{theorem}[\textcolor{red}{Generalized Law of Quadratic Reciprocity}]
Let $a$ and $b$ be odd positive integers.
    \begin{align*}
        \legendre{-1}{b}&= 
        \begin{cases}
            1 & \textnormal{if $b\equiv 1 \mod{4}$,} \\
            -1 & \textnormal{if $b \equiv 3 \mod{4}$.}
        \end{cases}
        \\ \legendre{2}{b}&=
        \begin{cases}
            1 & \textnormal{if $b\equiv 1$ or $7 \mod{8}$,} \\
            -1 & \textnormal{if $b \equiv 3$ or $5 \mod{8}$.}
        \end{cases}
        \\ \legendre{a}{b}&=
        \begin{cases}
            \legendre{a}{b} & \textnormal{if $a\equiv 1 \mod{4}$ or $b \equiv 1 \mod{4}$,} \\
            -\legendre{a}{b} & \textnormal{if $a \equiv b \equiv 3 \mod{4}$ and $q \equiv 3\mod{4}$.}
        \end{cases}
    \end{align*}
\end{theorem}

\setcounter{section}{23}
\section{What Primes are Sums of Two Squares?}

\begin{theorem}[\textcolor{red}{Sum of Two Squares Theorem for Primes}]
    Let $p$ be a prime. Then $p$ is a sum of two squares exactly when 
    \[p \equiv 1 \mod{4} \quad (\textnormal{or $p=2$)}.\]
\end{theorem}

\begin{comment}
\begin{statement}If $p$ is a sum of two squares, then $p \equiv 1 \mod{4}$.
\end{statement}


\begin{statement}If $p \equiv 1 \mod{4}$, then $p$ is a sum of two squares.
\end{statement}
\end{comment}

\begin{algorithm*}[\textcolor{red}{Method of Descent}] 
    Let $p$ be prime $\equiv 1 \mod{4}$.
    \begin{enumerate}[label=(\roman*),leftmargin=*]
        \item Given $A^2 + B^2 = Mp$ with $1<M<p$.
        \item Choose numbers $u$ and $v$ with 
            \[u \equiv A \mod{M} \quad \textnormal{and} \quad v \equiv B \mod{M},\]
            where $-\tfrac{M}{2} \leq u,v \leq \tfrac{M}{2}$.
        \item Find $1 \leq r < M$ such that $r=\frac{u^2+v^2}{M}$.    
    \item If $r=1$, conclude that 
            \[\left(\frac{uA+vB}{M}\right)^2 + \left(\frac{vA-uB}{M}\right)^2 = p;\]
            otherwise, write
            \[\left(\frac{uA+vB}{M}\right)^2 + \left(\frac{vA-uB}{M}\right)^2 = rp\]
            and repeat.
    \end{enumerate}
\end{algorithm*}


\setcounter{section}{26}
\section{Euler's Phi Function and Sums of Divisors}

\begin{definition*}
    The function $F:\Z \to \Z$ is defined by
    \[F(n)= \sum_{d \mid n} \varphi(d).\]
\end{definition*}

\begin{lemma}If $\gcd(m,n)=1$, then $F(mn)=F(m)F(n)$.
\end{lemma}

\begin{theorem}[\textcolor{red}{Euler's Phi Function Summation Formula}]
Let $n\in \Z$. Then
    \[F(n)=n.\]
\end{theorem}


\section{Powers Modulo $P$ and Primitive Roots}

\begin{definition*}[\textcolor{red}{Order of $a$ modulo $p$}]
    Let $a$ be an integer not divisible by the prime $p$. Then \textit{order of $a$ modulo $p$}, denoted $e_p(a)$, is the least positive exponent $e$ such that $a^e \equiv 1 \mod{p}$. 
\end{definition*}

\begin{theorem}[\textcolor{red}{Order Divisibility Property}]
    Let $a$ be an integer not divisible by the prime $p$, and suppose that $a^n \equiv 1 \mod{p}$. Then the order $e_p(a)$ divides $n$. In particular, the order $e_p(a)$ always divides $p-1$. 
\end{theorem}

\begin{definition*}[\textcolor{red}{Primitve Root modulo $p$}]
    A number $g$ with maximum order $e_p(g)=p-1$ is called a \textit{primitive root modulo $p$}.
\end{definition*}

\begin{theorem}[\textcolor{red}{Primitve Root Theorem}]
    Every prime $p$ has a primitive root. More precisely, there are exactly $\varphi(p-1)$ primitive roots modulo $p$.
\end{theorem}

\begin{definition*}
    Define $\psi: \N \to \N$ by
    \[\psi(d)=\#\set{a: 1 \leq a \leq p \ \textnormal{and} \ e_p(a)=d} .\]
\end{definition*}

\begin{proposition*}
    If $n$ divides $p-1$, then the congruence
    \[X^n -1 \equiv 0 \mod{p}\]
    has exactly $n$ solutions with $0 \leq X < p$.
\end{proposition*}

\end{document}
