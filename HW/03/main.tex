\documentclass[10pt,twoside]{amsart}

\usepackage[T1]{fontenc} 
\usepackage{geometry}
\usepackage{fancyhdr}
\usepackage{enumitem}
\usepackage{verbatim}
\usepackage{parskip}
% \usepackage{xcolor}
\usepackage{array} 
\usepackage{url} 
\usepackage{hyperref}
% \usepackage{float}
\usepackage{braket}
\usepackage{listings}
\usepackage{inconsolata}
\usepackage{mystyle}
\usepackage{lastpage}
% \usepackage{appendix}
% \usepackage{fontspec}

\begin{document}
\title{Homework 3\\\textnormal{Chris Powell\\Stardate 2019.44} }
\email{powel054@cougars.csusm.edu}
\maketitle
\thispagestyle{firststyle}

\begin{exercises}
\item Silverman 5.4. A number $L$ is called a common multiple of $m$ and $n$ if both $m$ and $n$ divide $L$. The smallest such $L$ is called the \textit{least common multiple} of $m$ and $n$ and is denoted by $\lcm(m,n)$.

        \begin{enumerate}[label=(\alph*)]
        \item Find the following lease common multiples.

            
\framebox{\begin{minipage}{\dimexpr\linewidth-2\fboxrule-2\fboxsep}
            \begin{enumerate}[label=(\roman*)]
                \item $\lcm(8,2) = 24$
                \item $\lcm(20,30)  = 60$
                \item $\lcm(51,68) = 204$
                \item $\lcm(23,18) = 414$
            \end{enumerate}
\end{minipage}}


        \item For each of the following LCMs that you computed in (a), compare the value of $\lcm(m,n)$ to the values of $m$, $n$ and $\gcd(m,n)$ Try to find a relationship.
        
\framebox{\begin{minipage}{\dimexpr\linewidth-2\fboxrule-2\fboxsep}
            \begin{enumerate}[label=(\roman*), itemsep=2mm]
                \item $\lcm(8,12) = 3 \cdot 8 = 2\cdot 12 = (3\cdot 2) \cdot \gcd(8,12) = (3\cdot 2) \cdot 4$

                    $\lcm(8,12) \cdot \gcd(8,12) = 24 \cdot 4 
                        = (8 \cdot 3)\cdot 4
                        = 8 \cdot (3\cdot 4)
                        = 8 \cdot 12$

                    \item $\lcm(20,30) = 3 \cdot 20 = 2 \cdot 30 = (3\cdot 2) \cdot \gcd(20,30) = (3\cdot 2) \cdot 10$
                    
                        $\lcm(20,30)\cdot \gcd(20,30) = 60\cdot 10 = (20 \cdot 3)\cdot 10 = 20 \cdot (3 \cdot 10) = 20 \cdot 30$  
                
                    \item $\lcm(51,68) = 4 \cdot 51=3 \cdot 68= (4 \cdot 3) \cdot \gcd(51,68) = (4 \cdot 3) \cdot 17$

                        $\lcm(51,68) \cdot \gcd(51,68) = 204 \cdot 17 = (51 \cdot 4) \cdot 17=51 \cdot (4 \cdot 17) = 51 \cdot 68$

                
                    \item $\lcm(23,18) = 18 \cdot 23 = 23 \cdot 18 = 23 \cdot 18 \cdot \gcd(23,18)= (23 \cdot 18) \cdot 1$
                        $\lcm(23,18) \cdot \gcd(23,18) = 414 \cdot 1 = (23 \cdot 18) \cdot 1= 23 \cdot (18 \cdot 1) = 23 \cdot 18$
                            
            \end{enumerate}
\end{minipage}}
         
        \item Give an argument proving that the relationship you found is correct for all $m$ and $n$.
      
\framebox{\begin{minipage}{\dimexpr\linewidth-2\fboxrule-2\fboxsep}
    \begin{proposition*} For all integers $m$ and $n$, 
        \[\lcm(m,n) \cdot \gcd(m,n)=mn.\]
    \end{proposition*}
    
    \begin{proof}
        Let $m,n \in \Z$. Assume $d = \gcd(m,n)$. Then $m = xd$ and $n = yd$ for some relatively prime integers $x$ and $y$. But $\ell = \lcm(m,n)$ is divisible by $m=xd$ and $n=yd$. So $\ell = dxy$ since $\gcd(x,y)=1$. Therefore,
        \[\lcm(m,n)\cdot \gcd(m,n) = d(dxy) = (dx)(dy) = mn.\]
    \end{proof}
\end{minipage}}        
        
        \item Use your result in (b) to compute $\lcm(301337,307829)$.


\framebox{\begin{minipage}{\dimexpr\linewidth-2\fboxrule-2\fboxsep}
            \begin{align*}
                \gcd(301337,307829)&= \gcd()
                \\&=541 
            \end{align*}
Therefore, by the above result,
           \begin{align*}
               \lcm(301337,307829)&=\frac{301337 \cdot 307829}{\gcd(301337,307289)} 
               \\&= \frac{301337 \cdot 307829}{541}
               \\&= \frac{92760267373}{541}
               \\&=1714607573
            \end{align*}
\end{minipage}}

        \item Suppose that $\gcd(m,n) = 18$ and $\lcm(m,n) = 720$. Find $m$ and $n$. Is there more than one possibility? Is so, find them all. 
    \end{enumerate}

\item Silverman 6.1. 
    
    \begin{enumerate}[label=(\alph*)]
        \item Find a solution in integers to the equation
            \[12345x+67890y=\gcd(12345,67890).\]

\framebox{\begin{minipage}{\dimexpr\linewidth-2\fboxrule-2\fboxsep}


    To evaluate the $\gcd(12345,67890)$, we apply the euclidean algorithm: 
            \begin{align*}
                67890 &= q_1(12345)+r_1   & (q_1,r_1) &= (5,6165)
                \\12345 &= q_2(6165)+r_2  & (q_2,r_2) &= (2,15)
                \\6165 &= q_3(15) + r_3   & (q_3,r_3) &= (411,0)
            \end{align*}
            
            Since $r_3=0$, we have $\gcd(12345,67890)=r_2=15$. Write $a=67890$ and $b=12345$. Then
            \begin{align*}
                r_1 &= a-q_1b
             \\     &= a-5b
            \\  r_2 &= b-r_1q_2 
                \\  &= b-(a-5b)(2)
                \\  &= -2a+11b
            \end{align*}
            So $(x,y)=(11,-2)$ is a solution. The set of all integer solutions is given by
    \[ \set{ (x,y) \in \Z^2 \mid x=11+k\left(\tfrac{12345}{15}\right), y=-2 - k \left(\tfrac{67890}{15}\right), k \in \Z}.\]
\end{minipage}}


        \item Find a solution in integers to the equation
            \[54321x+9876y=\gcd(54321,9876).\]
\framebox{\begin{minipage}{\dimexpr\linewidth-2\fboxrule-2\fboxsep}


    To evaluate the $\gcd(54321,9876)$, we apply the euclidean algorithm: 
    \begin{align*}
        54321 &= 9876(5)+4941 & (q_1,r_1) &= (5,4941)
    \\  9876  &= 4941(1)+4935 & (q_2,r_2) &= (1,4935)
    \\  4941  &= 4935(1)+6    & (q_3,r_3) &= (1,6)
    \\  4935  &= 6(822)+3     & (q_4,r_4) &= (822,3)
    \\  822   &= 3(274)+0     & (q_5,r_5) &= (284,0)
    \end{align*}
    Since $r_5=0$, we have $\gcd(54321,9876)=r_4=3$. Write $a=54321$ and $b=9876$. Then
    \begin{align*}
        r_1 &= a- q_1b    
        \\  &= a-5b
      \\r_2 &= b-r_1q_2   
        \\  &= b-(a-5b)(1) 
        \\  &= -a+6b
    \\  r_3 &= r_1 -r_2q_3  
         \\ &= (a-5b) - (-a+6b)(1) 
         \\ &= 2a+11b
    \\  r_4 &= r_2 - r_3q_4
        \\  &= (-a+6b) - (2a+11b)(822)
        \\  &= -1645a+9048b
    \end{align*}
Thus $(x,y)=(-1645,9048)$ is a solution to the given equation. Furthermore, the set of all integer solutions is given by
    \[ \set{ (x,y) \in \Z^2 \mid x=-1645+k\left(\tfrac{9876}{3}\right), y=9048 - k \left(\tfrac{54321}{3}\right), k \in \Z}.\]

\end{minipage}}

    
    
    \end{enumerate}

\begin{comment}

    \item Silverman 6.3. To prove the program works, show by induction that $b$
divides both $g_n - ax_n$ and $w_n − av_n$. (Incidentally, in Python you don't
need either of the variables $s$ or $t$.)

\begin{enumerate}[label = (\alph*)]
    
    \item Show that the algorithm described in Figure 6.1 computes the greatest common divisor of $g$ of the positive integers $a$ and $b$, together with the solution $(x,y)$ in integers to the equation $ax+by=\gcd(a,b)$

    \item Implement the algorithm on a computer using the computer language of your choice.

    \item Use your program to compute $g= \gcd(a,b)$ and integer solutions to $ax+by=g$ for the following pairs $(a,b)$.

        \begin{enumerate}[label=(\roman*)]

            \item $(19789,23548)$

            \item $(31875,8387)$

            \item $(22241739,19848039)$

        \end{enumerate}

            \item What happenso to your program if $b=0$? Fix the program so that it deals with this case correctly.

            \item For later applications it is uselful to have a solution with $x>0$. Modify your program so that it always returns a solution $x>0$.


\end{enumerate}


\end{comment}


\item Silverman 6.4. 

    \begin{enumerate}[label=(\alph*)]
        \item Find integers $x$, $y$, and $z$ that satisfy the equation
            \[6x+15y+20z=1.\]

\framebox{\begin{minipage}{\dimexpr\linewidth-2\fboxrule-2\fboxsep}
        Observe that    
        \begin{align*}
                20 &= 6(0) + 15(0) + 20(1)
                \\15 &= 6(0) + 15(1) + 20(0) 
                \\6  &= 6(1) + 15(0) + 20(0)
                \\5  &= 6(0) + 15(-1)+ 20(1)
                \\1  &= 6(1) + 15(1) + 20(-1)
            \end{align*}
            Therefore, $(x,y,z)=(1,1,-1)$ is a solution.
\end{minipage}}


        \item Under what conditions on $a$, $b$, and $c$ is it true that the equation
            \[ax+by+cz=1\]
            has a solution? Describe a general method of finding a solution when one exists.

\framebox{\begin{minipage}{\dimexpr\linewidth-2\fboxrule-2\fboxsep}
    The equation $ax+by+cz=1$ has integer solutions when $\gcd(\gcd(a,b),c)=1$. So first find a solution to 
    \[ax+by=\gcd(a,b).\]
    Then find a solution to 
        \[\gcd(a,b)k+cz=1.\]
\end{minipage}}

        \item Use your method from (b) to find a solution in integers to the equation
            \[155x+341y+385z=1.\]

    \end{enumerate}

\item Silverman 7.1. Suppose that $\gcd(a,b)=1$, and suppose further $a$ divides the product $bc$. Show that $a$ must divide $c$.

\framebox{\begin{minipage}{\dimexpr\linewidth-2\fboxrule-2\fboxsep}
    \begin{proof}
        As $\gcd(a,b)=1$, we know $ax+by=1$ for some $x,y \in \Z$. So $c(ax+by)=c(1)$. But $c = c(1)$ and 
        \[c(ax+by)=c(ax)+c(by)=(ca)x+(cb)y=(ac)x+(bc)y\] 
        since $\Z$ is a commutative ring with unity. But $a \mid bc$ by hypothesis, so $bc = ak$ for some $k \in \Z$. Thus
        \[c =(ac)x+(ak)y=a(cx)+a(ky)=a(cx+ky).\]
        Therefore, $a \mid c$.
    \end{proof}
\end{minipage}}

\item Silverman 7.2. Suppose that $\gcd(a,b)=1$, and suppose further that $a$ divides $c$ and that $b$ divides $c$. Show that the product $ab$ must divide $c$.

\framebox{\begin{minipage}{\dimexpr\linewidth-2\fboxrule-2\fboxsep}
    \begin{proof}
        As $\gcd(a,b)=1$, we can find $x,y \in \Z$ such that $ax+by=1$. So $c(ax+by)=c(1)$. But $c= c(1)$ and $c(ax+by)=c(ax)+c(by)$ since $\Z$ is a ring. By hypothesis, $a,b \mid c$. So $c=ka$ and $c = \ell b$ for some $k, \ell \in \Z$. Thus 
        \[c= (b \ell)(ax)+(ak)(by) = (ab)(\ell x)+ (ab)(ky) = ab(\ell x+ky).\]
        Hence, $c \mid ab$.
    \end{proof}
\end{minipage}}

\item Two players play the following game: The numbers 25 and 36 are written
on a board. On each player's turn, they select any two numbers currently
up and write on the board the positive difference of those two numbers,
provided the difference is not already written on the board. The game
continues until a player cannot write anything down. The last player to
write down a number wins. Does the game always end? If so, assuming
perfect play, who wins?


\framebox{\begin{minipage}{\dimexpr\linewidth-2\fboxrule-2\fboxsep}
    
Let $t$ be the total number of plays. Then $t \leq \max\set{36,25}-2=34$ since all plays are positive and the game begins with the numbers $36$ and $25$ already written. Since $\gcd(36,25)=1$, we can find some numbers which differ by $1$. So every $x \in S'= S \setminus \set{36,25}$ is a possible play. Thus $t \geq \abs{S'}=34$. Hence $t=34$. Now since the last player to write down a number wins, and $t$ is even, we know that the player who takes the second turn will win.
\end{minipage}}

\end{exercises}
\end{document}
