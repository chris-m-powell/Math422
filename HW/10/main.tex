\documentclass[10pt]{article}

% \usepackage[utf8]{inputenc}
\usepackage{geometry}
\usepackage{fancyhdr}
\usepackage{amsmath,amssymb,amsthm}
\usepackage{mathrsfs}
\usepackage{enumitem}
\usepackage{verbatim}
\usepackage{parskip}
\usepackage{xcolor}
\usepackage{array} 
\usepackage{url} 
\usepackage{float}
\usepackage{braket}
\usepackage{listings}
\usepackage{inconsolata}
\usepackage{mystyle}
\usepackage{lastpage}
\usepackage{fontspec}
\usepackage[euler-digits,euler-hat-accent]{eulervm}


\raggedbottom
\setmainfont{URW Palladio L}

\begin{document}
\title{\vspace{-2em}Homework 10\vspace{-1em}}
\author{Chris Powell}
\date{}
\maketitle
\thispagestyle{fancy}
\begin{enumerate}[itemsep=1em,label=\Alph*.,leftmargin=*]
    % \setlength{\leftmargini}{.0cm}
    \item Silverman 20.3. A number $a$ is called a \emph{cubic residue modulo $p$} if it is congruent to a cube modulo $p$, that is, if there is a number $b$ such that $a \equiv b^3 \mod{p}$.
            \begin{enumerate}[itemsep=1em, label=(\alph*), leftmargin=*]
            \item Make a list of all the cubic residues modulo $5$, modulo $7$, modulo $11$, and modulo $13$.
            
                \framebox{\begin{minipage}{\dimexpr\linewidth-2\fboxrule-2\fboxsep}
                Observe that
                    \begin{align*}
                        \eq{0^3}{5} &= \eq{0}{5}^3 = \eq{0}{5} & \eq{0^3}{7} &= \eq{0}{7}^3 = \eq{0}{7} \\
                        \eq{1^3}{5} &= \eq{1}{5}^3 = \eq{1}{5} & \eq{1^3}{7} &= \eq{1}{7}^3 = \eq{1}{7} \\
                        \eq{2^3}{5} &= \eq{8}{5} = \eq{3}{5}  & \eq{2^3}{7} &= \eq{8}{7} = \eq{1}{7} \\
                        \eq{3^3}{5} &= \eq{27}{5} = \eq{2}{5}  &  \eq{3^3}{7} &= \eq{27}{7} = \eq{6}{7} \\
                        \eq{4^3}{5} &= \eq{64}{5} = \eq{4}{5} & \eq{4^3}{7} &= \eq{64}{7} = \eq{1}{7} \\
                        && \eq{5^3}{7} &= \eq{125}{7} = \eq{6}{7}\\
                        && \eq{6^3}{7} &= \eq{216}{7} = \eq{6}{7}
                    \end{align*}
                        \begin{align*}
                            \eq{0^3}{11} &= \eq{0}{11}^3 = \eq{0}{11} & \eq{0^3}{13} &= \eq{0}{13}^3 = \eq{0}{13}\\
                            \eq{1^3}{11} &= \eq{1}{11}^3 = \eq{1}{11} & \eq{1^3}{13} &= \eq{1}{13}^3 = \eq{1}{13}\\
                            \eq{2^3}{11} &= \eq{8}{11} = \eq{8}{11} & \eq{2^3}{13} &= \eq{8}{13} = \eq{8}{13}\\
                            \eq{3^3}{11} &= \eq{27}{11} = \eq{5}{11} & \eq{3^3}{13} &= \eq{27}{13} = \eq{1}{13}\\
                            \eq{4^3}{11} &= \eq{64}{11} = \eq{9}{11} & \eq{4^3}{13} &= \eq{64}{13} = \eq{12}{13}\\
                            \eq{5^3}{11} &= \eq{125}{11} = \eq{4}{11} & \eq{5^3}{13} &= \eq{125}{13} = \eq{4}{13}\\
                            \eq{6^3}{11} &= \eq{216}{11} = \eq{7}{11} & \eq{6^3}{13} &= \eq{216}{13} = \eq{8}{13}\\
                            \eq{7^3}{11} &= \eq{343}{11} = \eq{2}{11} & \eq{7^3}{13} &= \eq{343}{13} = \eq{5}{13}\\
                            \eq{8^3}{11} &= \eq{512}{11} = \eq{6}{11} & \eq{8^3}{13} &= \eq{512}{13} = \eq{8}{13}\\
                            \eq{9^3}{11} &= \eq{729}{11} = \eq{3}{11} & \eq{9^3}{13} &= \eq{729}{13} = \eq{1}{13}\\
                            \eq{10^3}{11} &= \eq{1000}{11} = \eq{10}{11} & \eq{10^3}{13} &= \eq{1000}{13} = \eq{12}{13}\\
                            &&\eq{11^3}{13} &= \eq{1331}{13} = \eq{5}{13}\\
                            &&\eq{12^3}{13} &= \eq{1728}{13} = \eq{12}{13}
                    \end{align*}
                        So $a \in \set{0,1,2,3,4}$ if $p=5$, $a \in \set{0,1,6}$ if $p=7$, $a \in \set{0,\ldots,10}$ if $p=11$, and $a \in \set{0,1,5,8,12}$ if $p=13$.
\end{minipage}}

\item Find two numbers $a_1$ and $b_1$ such that neither $a_1$ nor $b_1$ is a cubic residue modulo $19$, but $a_1b_1$ is a cubic residue modulo $19$. Similarly, find two numbers $a_2$ and $b_2$ such that none of the three numbers $a_2,b_2$, or $a_2b_2$ is a cubic residue modulo $19$.
                \vspace{1em}
                    \begin{lstlisting}
def residue(n, p):
    """Return each a cong b^n mod p for some integer b and modulus p; n=2,3"""
    R = []
    for i in range(p):
        R += [(i**n) % p]
    return set(R)
                    \end{lstlisting}
                    \begin{proof}
                        It is clear that the program terminates since it iterates via a \texttt{for} loop. Let $R_i$ be the value of the list $\texttt{R}$ after the $i$\textsuperscript{th} iteration. Assume the value of the parameter $\texttt{n}$ is $3$. At each iteration, $R_i$ is a collection of integers $x_0,\ldots,x_{i-1}$ such that $\eq{x_k}{p}=\eq{i_k^n}{p}$ for some $i_k \in \Z$ where $i \in \set{0,\ldots,p-1}$. Therefore, by definition, $R_i$ is a list of cubic residues modulo $p$. The program returns $\texttt{set(R)}$, giving all distinct elements of in the list $R_{p-1}$. This concludes the proof of correctness. 
                    \end{proof}
                    
                \vspace{1em}
                \framebox{\begin{minipage}{\dimexpr\linewidth-2\fboxrule-2\fboxsep}
                By the above algorithm, the set of cubic residues modulo $19$ is
                    \[R=\set{0, 1, 7, 8, 11, 12, 18}.  \]
                    Consider $a_1=3$ and $b_1=9$. Then 
                    \[\eq{a_1}{19}\cdot \eq{b_1}{19} = \eq{3}{19}\cdot \eq{9}{19} =\eq{3\cdot 9}{19} = \eq{27}{19}=\eq{8}{19}.\] Note that $8 \in S$, yet neither $3$ nor $9$ are in $R$.
                Next, consider $a_2=4$ and $b_2=6$. Then
                    \[\eq{a_1}{19}\cdot \eq{b_1}{19} = \eq{4}{19}\cdot \eq{6}{19} =\eq{4\cdot 6}{19} = \eq{24}{19}=\eq{5}{19}.\] 
                    But $4,5,6 \notin R$.
                \end{minipage}}
           \pagebreak 
            \item If $p\equiv 2 \mod{3}$, make a conjecture as to which $a$'s are cubic residues. Prove that your conjecture is correct.  

                \framebox{\begin{minipage}{\dimexpr\linewidth-2\fboxrule-2\fboxsep}
                    \begin{conjecture*}
                        If $p \equiv 2 \mod{3}$, then every integer not divisible by $p$ is a cubic residue modulo $p$.
                    \end{conjecture*}
                    \begin{proof}
                        Let $a \in \Z$ such that $p \nmid a$. Assume $\eq{p}{3}=\eq{2}{3}$. Then $3 \mid p-2$. So $p-2=3k$ for some $k \in \Z$. Thus $p=3k+2$ which implies $p-1=3k+1$. Since $p$ is prime and $\eq{a}{p}\neq \eq{0}{p}$, Fermat's Little Theorem implies $\eq{a^{3k+1}}{p}=\eq{1}{p}$. It follows that $\eq{a^{3k+2}}{p}=\eq{a}{p}$. Thus
                        \begin{align*}
                            \eq{a}{p} &= \eq{a}{p} \cdot \eq{1}{p}
                            \\&= \eq{a^{3k+1}}{p} \cdot \eq{a^{3k+2}}{p}
                            \\&= \eq{a^{3k+1} \cdot a^{3k+2}}{p}
                            \\&= \eq{a^{6k+3}}{p}
                            \\&= \eq{a^{3(2k+1)}}{p}
                            \\&= \eq{\left(a^{ (2k+1)} \right)^3}{p}.
                        \end{align*}
                    \end{proof}
                \end{minipage}}
            \end{enumerate}

            \pagebreak
    \item  Suppose that $p$ is a prime with $p \equiv 1 \mod{3}$. Let $a \in \Z$ with $p \nmid a$.
        \begin{enumerate}[label=(\alph*),leftmargin=*]
            \item Show that if $a$ is a cubic residue, then $a^{(p−1)/3} \equiv 1 \mod{p}$. 

                \framebox{\begin{minipage}{\dimexpr\linewidth-2\fboxrule-2\fboxsep}
                    \begin{proof}
                    Suppose $\eq{a}{p}=\eq{b^3}{p}$ for some $b \in \Z$. Then 
                    \[\eq{a^{(p-1)/3}}{p} = \eq{\left(b^3\right)^{(p-1)/3}}{p}=\eq{b^{p-1}}{p}.\]
                    Since $\eq{a}{p} \neq \eq{0}{p}$, and $\Z_p$ is an integral domain, the zero-product property implies $\eq{b^3}{p}=\eq{b}{p}^3 \neq \eq{0}{p}$
                    Therefore, since $p$ is prime, it follows from Fermat's Little Theorem that $\eq{b^{p-1}}{p}=\eq{1}{p}$. Hence, $\eq{a^{(p-1)/3}}{p} = \eq{1}{p}$.
                    \end{proof}
                \end{minipage}}
            
        
        \end{enumerate}

            \pagebreak
    \item Write a program that implements the CRT for an arbitrary list of moduli.
The input should be a list of ordered pairs $[(a_1,m_1),(a_2,m_2),\ldots,(a_n,m_n)]$
where the $m_i$ are pairwise relatively prime, and the output should be a
        such that $a \equiv a_i \mod{m_i}$ for all $i$. Remember to prove your algorithm works!
\vspace{1em}
                \begin{lstlisting}
def xgcd(a, b):
    """Return (g, x, y) such that a*x + b*y = g = gcd(a, b)"""
    if b == 0:
        return a, 1, 0
    x, g, v, w = 1, a, 0, b
    while w != 0:
        x, g, v, w = v, w, x - (g // w) * v, g % w
    x = x % (b // g)
    return g, x, (g - (a * x)) // b

                
def CRT(L):
    """Given (a1,m1),...,(an,mn) with gcd(mi,mj)=1, return x such that x = ai (mod mi) for all i"""
    x, M = 0, 1
    for i in range(len(L)):
        M *= L[i][1]
    for i in range(len(L)):
        x += L[i][0] * (M // L[i][1]) * (xgcd(M//L[i][1], L[i][1])[1] % L[i][1])
    return x % M
                \end{lstlisting}
                    \begin{proof}Termination must occur since the program iterates via $\texttt{for}$. I show correctness. The input is a list of ordered pairs $(a_1,m_1),\ldots, (a_n,m_n)$, where
                        the $m_i$ are pairwise relativey prime. Let $M$ be the value of $\texttt{M}$ after completion of the first \texttt{for} loop. Then clearly, $M=\prod_{k=1}^{n}m_k$. Now consider the program's execution after completion of the first $\texttt{for}$ loop.
                        Let $x_i$ be the value of $\texttt{x}$ after the $i\textsuperscript{th}$ iteration. Recall that for relatively prime integers $a$ and $b$, $\texttt{xgcd(a,b)}$ returns $\left(\gcd(a,b),u,v\right)$, where $\eq{u}{b}=\eq{a}{b}^{-1}$ and $\eq{v}{b}=\eq{b}{a}^{-1}$. The program returns 
                        \[\eq{x_n}{M} = \eq{\sum_{i=1}^{n} \left( a_{i} \cdot \frac{M}{m_{i}} \cdot \eq{ \frac{M}{m_{i}}}{m_{i}}^{-1}  \right)}{M}. \]
                    Therefore, by the Generalized Chinese Remainder Theorem, the program gives the correct output.
                    \end{proof}

            \pagebreak
\item Let $f(x)$ be a polynomial, and suppose $m, n \in N$ with $\gcd(m,n) = 1$. Show that $f(x) \equiv 0 \mod{mn}$ has a solution if and only if $f(x) \equiv 0 \mod{m}$ and $f(x) \equiv 0 \mod{n}$ both have solutions.


                \framebox{\begin{minipage}{\dimexpr\linewidth-2\fboxrule-2\fboxsep}
                    \begin{proof}
                        Suppose $\eq{f(a)}{m}=\eq{0}{m}$ and $\eq{f(b)}{n}=\eq{0}{n}$ for some $a,b \in \Z$. Since $\gcd(m,n)=1$, the Chinese Remainder Theorem implies there is a unique $c \in \Z_{mn}$ such that $\eq{c}{m} = \eq{a}{m}$ and $\eq{c}{n}=\eq{b}{n}$. Thus $\eq{f(c)}{m}=\eq{0}{m}$ and $\eq{f(c)}{n}=\eq{0}{n}$. Applying the Chinese Remainder Theorem again, we that $\eq{f(c)}{mn}=\eq{0}{mn}$. Conversely, assume $\eq{f(c)}{mn}=\eq{0}{mn}$. Then $mn \mid f(x)$. So $f(x)=mnk$ for some $k \in \Z$. By multiplicative associativity and commutativity of $\Z$,
                        \[f(x)=m(nk)=n(mk).\]
                        Thus, by multiplicative closure of $\Z$, $m,n \mid f(x)$. Hence, $\eq{f(x)}{m}=\eq{0}{m}$ and $\eq{f(x)}{n}=\eq{0}{n}$.
                    \end{proof}

                \end{minipage}}

            \pagebreak
\item 
    \begin{enumerate}[label=(\alph*)]
        \item Find all solutions to $x^2 \equiv 1 \mod{143}$ using the Chinese Remainder Theorem.
        
        
                \framebox{\begin{minipage}{\dimexpr\linewidth-2\fboxrule-2\fboxsep}
               
                    The composite modulus $143$ has factorization $143=11*13$, where $11$ and $13$ are distinct primes. Consider the congruences 
                    \[\eq{x^2}{11}=\eq{1}{11} \quad \textnormal{and} \quad \eq{x^2}{13}=\eq{1}{13}.\]
                    Observe that 
                    \begin{align*}\eq{x^2}{11} = \eq{1}{11} &\Leftrightarrow \eq{x^2}{11}-\eq{1}{11}=\eq{0}{11}
                            \\&\Leftrightarrow \eq{x^2-1}{p}=\eq{0}{11}
                            \\&\Leftrightarrow \eq{(x+1)(x-1)}{11}=\eq{0}{11}.
                        \end{align*}
                    As $11$ is prime, $\Z_{11}$ is an integral domain. So by the zero-product property, either
                    \[ \eq{x+1}{11}=\eq{0}{11} \quad \textnormal{or} \quad \eq{x-1}{11}=\eq{0}{11}.\]
                    So either
                    \[\eq{x}{11}=-\eq{1}{11}=\eq{-1}{11}=\eq{10}{11} \quad \textnormal{or} \quad \eq{x}{11}=\eq{1}{11}.\] 
                    Similarly, either 
                    \[ \eq{x+1}{13}=\eq{0}{13} \quad \textnormal{or} \quad \eq{x-1}{11}=\eq{0}{13}.\]
                    So either 
                    \[\eq{x}{13}=-\eq{1}{13}=\eq{-1}{13}=\eq{12}{13} \quad \textnormal{or} \quad \eq{x}{13}=\eq{1}{13}.\]
                    By the Chinese Remainder Theorem, there is a unique $x \in \Z_{143}$ such that
                    \begin{align*}
                    \begin{cases}
                        \eq{x}{11}=\eq{1}{11}\\
                        \eq{x}{13}=\eq{1}{13},
                    \end{cases} 
                    \begin{cases}
                        \eq{x}{11}=\eq{10}{11}\\
                        \eq{x}{13}=\eq{1}{13},
                    \end{cases}
                    \begin{cases}
                        \eq{x}{11}=\eq{1}{11}\\
                        \eq{x}{13}=\eq{12}{13},
                    \end{cases} 
                    \begin{cases}
                        \eq{x}{11}=\eq{10}{11}\\
                        \eq{x}{13}=\eq{12}{13}.
                    \end{cases}
                    \end{align*}
                    Using the $\texttt{CRT}$ program from Exercise C, I get solutions
                    \[ \eq{1}{143}, \eq{131}{143}, \eq{12}{143}, \eq{142}{143},\]
                    respectively.
                 \end{minipage}}
       \pagebreak 
        \item Let $p, q$ be distinct primes. How may solutions does $x^2 \equiv 1 \mod{pq}$ have?


                \framebox{\begin{minipage}{\dimexpr\linewidth-2\fboxrule-2\fboxsep}
                    \begin{proof}
                        Assume $p$ and $q$ are odd primes. Observe that
                        \begin{align*}\eq{x^2}{p} = \eq{1}{p} &\Leftrightarrow \eq{x^2}{p}-\eq{1}{p}=\eq{0}{p}
                            \\&\Leftrightarrow \eq{x^2-1}{p}=\eq{0}{p}
                            \\&\Leftrightarrow \eq{(x+1)(x-1)}{p}=\eq{0}{p}.
                        \end{align*}
                        Since $\Z_p$ is an integral domain, it follows from the zero-product property that either $\eq{x+1}{p}=\eq{0}{p}$ or $\eq{x-1}{p}=\eq{0}{p}$. But we cannot have both $\eq{x+1}{p}=\eq{0}{p}$ and $\eq{x-1}{p}=\eq{0}{p}$, as then $p \mid x+1, x-1$ which implies
                         \[p\mid (x+1)-(x-1)=2,\]
                         a contradiction since $p>2$. Now if $\eq{x+1}{p}=\eq{0}{p}$, then $\eq{x}{p}=\eq{-1}{p}$. 
                         On the other hand, $\eq{x-1}{p}=\eq{0}{p}$ implies that $\eq{x}{p}=\eq{1}{p}$.
                         Thus the congruence $\eq{x^2}{p}=\eq{1}{p}$ has exactly 2 solutions, namely, $\eq{\pm 1}{p}$; likewise $\eq{x^2}{q}=\eq{1}{q}$ has solutions $\eq{\pm 1}{q}$. By the Chinese Remainder Theorem, there is a unique $x \in \Z_{pq}$ such that $\eq{x}{p}=\eq{a}{p}$ and $\eq{x}{q}=\eq{b}{p}$ for each $\left(a,b\right) \in \set{\eq{\pm 1}{p}} \times \set{\eq{\pm 1}{q}}$. Hence the congruence $\eq{x^2}{p}=\eq{1}{p}$ has exactly $2^2=4$ solutions. If $p$ or $q$ is $2$, then there are exactly $2$ solutions since $\eq{1}{2}=\eq{-1}{2}$.
                        
                    \end{proof}
                \end{minipage}}


        \item Let $p_1,p_2,\ldots, p_r$ be distinct primes. How many solutions does $x^2 \equiv 1 \mod{p_1 p_2 \cdots p r}$ have?
    
    
                \framebox{\begin{minipage}{\dimexpr\linewidth-2\fboxrule-2\fboxsep}
                    \begin{proof}
                        Assume that each prime $p_i$ is odd. Let $m = \prod_{i=0}^r p_i$. Then by the Generalized Chinese Remainder Theorem, there is a unique $x\in \Z_m$ such that $\eq{x}{p_i}=\eq{a_i}{p_i}$ for all $i \in \set{1,\ldots,r}$. 
                        But by the argument given in part (b), each congruence $\eq{x^2}{p_i} = \eq{1}{p_i}$ has the solutions set $\set{\eq{\pm 1}{p_i}}$. In other words, there is exactly one $x \in \Z_m$ satisfying $\eq{x}{p_i}=\eq{a_i}{p_i}$ for each $(a_1,\ldots,a_r) \in \set{ \eq{\pm 1}{p_1}}\times \cdots \times \set{\eq{\pm 1}{p_r}}$. Thus $\eq{x^2}{m}=\eq{1}{m}$ has exactly $2^r$ solutions. If $p_i=2$ some for some $i$, then there are exactly $2^{r-1}$ solutions since $\eq{1}{2}=\eq{-1}{2}$
                    \end{proof}
                \end{minipage}}

     
    \end{enumerate}


\end{enumerate}
\end{document}
