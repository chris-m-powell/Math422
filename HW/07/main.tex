\documentclass[10pt,twoside]{article}

% \usepackage[utf8]{inputenc}
\usepackage{geometry}
\usepackage{fancyhdr}
\usepackage{amsmath,amssymb,amsthm}
\usepackage{mathrsfs}
\usepackage{enumitem}
\usepackage{verbatim}
\usepackage{parskip}
\usepackage{xcolor}
\usepackage{array} 
\usepackage{url} 
\usepackage{float}
\usepackage{braket}
\usepackage{listings}
\usepackage{inconsolata}
\usepackage{mystyle}
\usepackage{lastpage}
\usepackage{fontspec}
\usepackage[euler-digits,euler-hat-accent]{eulervm}


\raggedbottom
\setmainfont{URW Palladio L}
% \newfontfamily\headingfont[Path = /usr/share/fonts/opentype/bebas-neue/]{BebasNeue Regular.otf}
% \setmainfont[Path = /usr/share/fonts/opentype/tex-gyre-pagella/]{texgyrepagella-regular.otf}
\begin{document}
\title{\vspace{-2em}Homework 7\vspace{-1em}}
\author{Chris Powell}
\date{}
\maketitle
\thispagestyle{fancy}

\begin{enumerate}[itemsep=2em,label=\Alph*.]
    \item Write a program that takes as input a positive integer $n$ and computes $\varphi(n)$. You may use brute force.
        \begin{lstlisting}
def gcd(a, b):
    """Return gcd(a,b)"""
    while b:
        a, b = b, a % b
    return a
       
def totient(n):
    """Given positive integer n, return totient(n)"""
    t = 0
    for i in range(1, n):
        if gcd(i, n) == 1:
            t += 1
    return t
        \end{lstlisting}
    \framebox{\begin{minipage}{\dimexpr\linewidth-2\fboxrule-2\fboxsep}
        \begin{proof}
            Recall that termination and correctness of \texttt{gcd} has already been shown. Since \texttt{totient} iterates using \texttt{for}, it is clear that the algorithm must terminate. It remains to show that \texttt{totient} gives the correct output. Let $t_k$ be the value of \texttt{t} after $k$ iterations. If $\gcd(i,n)=1$, then $t_{k+1}=t_k +1$; otherwise, $t_{k+1}=t_{k}$. So 
            \[t_n= \sum_{\substack{1\leq i \leq n \\ gcd(i,n)=1}}i. \]       
            But this is the totient $\varphi(n)$, by definition.
            Therefore, since the algorithm returns $t_n$, it gives the correct output.
        \end{proof}
    \end{minipage}}

    \item Compute
        \begin{enumerate}[label=\arabic*.]
            \item $\varphi(81)$
   
                \framebox{\begin{minipage}{\dimexpr\linewidth-2\fboxrule-2\fboxsep}
            Observe that 
                \begin{align*}
                    \varphi(81)&=\varphi\left(3^4\right)
                    \\&=3^4-3^3 \justify{Theorem 11.1.a}
                    \\&=81-27
                    \\&=54.
                \end{align*}
    \end{minipage}} 

            \item $\varphi(20736)$
    
                \framebox{\begin{minipage}{\dimexpr\linewidth-2\fboxrule-2\fboxsep}
                Observe that
                \begin{align*}
                    \varphi(20736)&= \varphi \left(2^8 \cdot 3^4\right)
                    \\&=\varphi \left(2^8 \right) \varphi \left(3^4 \right) \justify{Theorem 11.1.b}
                    \\&=\left(2^8-2^7 \right) \left(3^4-3^3 \right) \justify{Theorem 11.1.a}
                    \\&=(256-128)(81-27)
                    \\&=(128)(54)
                    \\&=6912.
                \end{align*}
    \end{minipage}}

            \item $\varphi(10000000000)$
    
                \framebox{\begin{minipage}{\dimexpr\linewidth-2\fboxrule-2\fboxsep}
                Observe that
                \begin{align*}
                    \varphi(1000000000000)&=\varphi\left(10^{12}\right)
                    \\&=\varphi\left( (2\cdot5)^{12} \right)
                    \\&=(2\cdot 5)^{12} \cdot \left(1- \frac{1}{2}\right)\left(1- \frac{1}{5}\right)
                    \\&=(2\cdot 5)^{12} \cdot \frac{1}{2}\cdot \frac{4}{5}
                    \\&=(2\cdot 5)^{12} \cdot \frac{4}{10}
                    \\&=(2\cdot 5)^{11} \cdot 4
                    \\&=10^{11} \cdot 4
                    \\&=40^{11}.
                \end{align*}
    \end{minipage}}
        \end{enumerate}
    \item 
        \begin{enumerate}[label=\arabic*.]
            \item Find all $n$ for which $\varphi(n)=4$.
      
                \framebox{\begin{minipage}{\dimexpr\linewidth-2\fboxrule-2\fboxsep}
                Observe that
                    \begin{align*}
                    \varphi(n)&=\varphi\left( \prod_{i=1}^r p_i^{e_i}\right)
                    \\&=\prod_{i=1}^r \varphi\left( p_i^{e_i}\right)
                    \\&=\prod_{i=1}^r \left( p_i^{e_i}-p_i^{e_i-1}\right)
                    \\&=\prod_{i=1}^r \left( (p-1)p_i^{e_i-1}\right)
                \end{align*}
                So $(p-1) \mid \varphi(n)=4$. But 
                \[\set{p \in \N \mid p \textnormal{ is prime}, p-1 \textnormal{ divides } 4}=\set{2,3,5}.\]
                We consider the pairs of the powers of such of $p$:
                \begin{align*}
                    \varphi(10)&=\varphi(2^1 \cdot 5^1)=\varphi(2^1)\varphi(5^1)=(2^1-2^0)(5^1-5^0)=1\cdot 4 =4
                    \\ \varphi(12)&=\varphi(2^2 \cdot 3^1)=\varphi(2^2)\varphi(3^1)=(2^2-2^1)\cdot (3^1-3^0)  =2 \cdot 2 = 4
                    \\ \varphi(5)&=\varphi(2^0 \cdot 5^1)=5^1-5^0= 4.
                    \\ \varphi(8)&=\varphi(2^3 \cdot 3^0)=\varphi(2^3)\varphi(3^0)=(2^3-2^2)=8-4= 4.
                \end{align*}
                    We do not need to consider $3^e$ for $e \geq 2$ since $\varphi(3^2) = 3^2-3^1= 6 > 4$.
                    By similar reasoning, we do not need to consider $2^e$ for $e\geq 4$, nor $5^e$ for $e \geq 2$.
                Hence $\set{n \in \N \mid \varphi(n)=4}=\set{5,8,10,12}$.
                \end{minipage}}

            \item Find all $n$ for which $\varphi(n)=6$.
                
                \framebox{\begin{minipage}{\dimexpr\linewidth-2\fboxrule-2\fboxsep}
                Observe that
                \begin{align*}
                    \varphi(n)&=\varphi\left( \prod_{i=1}^r p_i^{e_i}\right)
                    \\&=\prod_{i=1}^r \varphi\left( p_i^{e_i}\right)
                    \\&=\prod_{i=1}^r \left( p_i^{e_i}-p_i^{e_i-1}\right)
                    \\&=\prod_{i=1}^r \left( (p-1)p_i^{e_i-1}\right)
                \end{align*}
                So $(p-1) \mid \varphi(n)=6$. But 
                \[\set{p \in \N \mid p \textnormal{ is prime}, p-1 \textnormal{ divides } 6}=\set{2,3,7}.\]
                We consider the pairs of the powers of such of $p$:
                \begin{align*}
                    \varphi(7)&=\varphi(2^0 \cdot 7^1)=7^1-7^0=6
                    \\ \varphi(9)&=\varphi(2^0 \cdot 3^2)=3^2-3^1=9-3=6
                    \\ \varphi(14)&=\varphi(2^1 \cdot 7^1)=\varphi(2^1)\varphi(7^1)=(2^1-2^0)(7^1-7^0)=1\cdot 6 = 6
                    \\ \varphi(18)&=\varphi(2^1 \cdot 3^2)=\varphi(2^1)\varphi(3^2)=(2^1-2^0)(3^2-3^1)=1\cdot 6 = 6
                \end{align*}
                We do not need to consider $3^e$ for $e \geq 3$, nor $7^e$ for $e \geq 2$ as their totient will be greater than $6$.
                Hence $\set{n \in \N \mid \varphi(n)=6}=\set{7,9,14,18}$.
                \end{minipage}}

        \end{enumerate}
            \item Silverman 11.5 For each part, find an $x$ that solves the given simultaneous congruences.
        \begin{enumerate}[label=(\alph*)]
            \item $x \equiv 3 \mod{7}$ and $x \equiv 5 \mod{9}$
            
    \framebox{\begin{minipage}{\dimexpr\linewidth-2\fboxrule-2\fboxsep}
        Since $\gcd(7,9)=1$, the Chinese Remainder Theorem implies there is a unique $x \in \Z / (7*9)\Z$
        satisyfing the given system of congruences. To find $x$, we apply the \texttt{crt} algorithm developed in exercise E. We obtain $x=\texttt{crt(3,5,7,9)}=59$.
    \end{minipage}}
                
    \item $x \equiv 3 \mod{37}$ and $x \equiv 1 \mod{87}$

    \framebox{\begin{minipage}{\dimexpr\linewidth-2\fboxrule-2\fboxsep}
        Since $\gcd(37,87)=1$, the Chinese Remainder Theorem implies there is a unique $x \in \Z / (37*87)\Z$
        satisyfing the given system of congruences. To find $x$, we apply the \texttt{crt} algorithm developed in the following exercise E. We obtain $x=\texttt{crt(3,1,37,87)}=262$.
    \end{minipage}}

    \item $x \equiv 5 \mod{7}$ and $x \equiv 2 \mod{12}$ and $x \equiv 8 \mod{13}$
        \end{enumerate}

        
    \item Silverman 11.8. You may not use brute force. Write a program that takes as input four integers $(b,c,m,n)$ with $\gcd(m,n)=1$ and computes an integer $x$ with $0 \leq x \leq mn$ satisfying
        \[ x \equiv b \mod{m} \quad \textnormal{and} \quad x \equiv c \mod{n}.\]
    
\begin{lstlisting}    
def xgcd(a, b):
    """Return (g, x, y) such that a*x + b*y = g = gcd(a, b)"""
    if b == 0:
        return a, 1, 0
    x, g, v, w = 1, a, 0, b
    while w != 0:
        x, g, v, w = v, w, x - (g // w) * v, g % w
    x = x % (b // g)
    return g, x, (g - (a * x)) // b


def crt(a, b, m, n):
    """Given integers a,b,m,n, with gcd(m,n)=1, return unique x cong a,b (mod m,n)"""
    (g, r, s) = xgcd(m, n)
    return ((a * s * n) + (b * r * m)) % (m * n)
\end{lstlisting}
    
    \framebox{\begin{minipage}{\dimexpr\linewidth-2\fboxrule-2\fboxsep}
   \begin{proof}
       Recall that the termination and correctness of \texttt{xgcd} has already been shown. It immediately follows that \texttt{crt} must terminate. We show that \texttt{crt} gives the correct output. By the correctness of $\texttt{xgcd}$, $r$ and $s$ are such that
       \[rm+sn=gcd(m,n)=1.\]
       This implies $rm \equiv 1\mod{n}$ and $sn \equiv 1 \mod{m}$. So $a(rm) \equiv a \mod{m}$ and $b(rm) \equiv b \mod{n}$. The algorithm returns $a(rn)+b(sm) \mod{mn}$. But
       \begin{align*}
           a(sn)+b(rm) \equiv a(sn) \equiv a(1) \equiv a \mod{m}
       \end{align*}
       and
       \begin{align*}
           a(sn)+b(rm) \equiv b(rm) \equiv b(1) \equiv b \mod{n}.
       \end{align*}
       We've already shown that when $\gcd(m,n)=1$, the map $[x]_{mn} \mapsto \left([x]_m,[x]_n\right)$ is a bijection $\Z/mn \to \Z/m \times \Z/n$. Hence, the algorithm returns the correct output.
   \end{proof}
    \end{minipage}}

    \item Silverman 11.9 Let $m_1,m_2,m_3$ be positive integers such that each pair is relatively prime. That is,
        \[\gcd(m_1,m_2)=1 \quad \textnormal{and} \quad \gcd(m_1,m_3)=1 \quad \textnormal{and} \quad \gcd(m_2,m_3)=1.\]
  Let $a_1,a_2,a_3$ be any three integers. Show that there is exactly one integer $x$ in the interval $0 \leq x < m_1m_2m_3$ that simultaneously solves the three congruences
        \[x \equiv a_1 \mod{m_1}, \quad x \equiv a_2 \mod{m_2}, \quad x \equiv a_3 \mod{m_3}.\]
Can you figure out how to generalize this problem to deal with lots of congruences
        \[x \equiv a_1 \mod{m_1}, \quad x \equiv a_2 \mod{m_2}, \ldots, \quad x \equiv a_r \mod{m_r}?\]
   In particular, what conditions do the moduli $m_1,m_2, \ldots,m_r$ need to satisfy? 
    \item Show that if $\gcd(m,n)>1$, then
        \begin{align*}
            \psi: \Z / mn &\to \Z /m \times \Z /n
            \\ [x] &\mapsto ([x],[x])
        \end{align*}
is never bijective.        

\framebox{\begin{minipage}{\dimexpr\linewidth-2\fboxrule-2\fboxsep}
\begin{proof}
    Assume $g=\gcd(m,n)>1$. Then $g \mid mn$ since $g \mid m,n$. So $mn=gd$ for some $d \in \Z$. Thus $d=\frac{m}{g}n= m\frac{n}{g}$. We know $\frac{m}{g},\frac{n}{g} \in \Z$ since $g \mid m,n$. Therefore $m,n \mid d$. So $d \equiv 0 \mod{m}$ and $d \equiv 0 \mod{n}$. So $\psi \left([d]_{mn}\right)=\left([0]_m,[0]_n\right)$. But $1< d < mn$ since $\gcd(m,n)>1$. So $[d]_{mn} \neq [0]_{mn}$. Hence, $\psi$ is not injective. 
\end{proof}
\end{minipage}}

\end{enumerate}

\end{document}
