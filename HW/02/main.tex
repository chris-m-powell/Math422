\documentclass[10pt,twoside]{amsart}

\usepackage[T1]{fontenc} 
\usepackage{geometry}
\usepackage{fancyhdr}
\usepackage{enumitem}
\usepackage{verbatim}
\usepackage{parskip}
\usepackage{xcolor}
\usepackage{array} 
\usepackage{url} 
\usepackage{hyperref}
\usepackage{float}
\usepackage{braket}
\usepackage{listings}
\usepackage{inconsolata}
\usepackage{mystyle}
\usepackage{lastpage}
\usepackage{appendix}

\begin{document}
\title{Homework 2\\\textnormal{Chris Powell\\Stardate 2019.37} }
\email{powel054@cougars.csusm.edu}
\maketitle
\thispagestyle{firststyle}

\begin{exercises}
    \item A circular dial has the numbers 0 through 56 inscribed at equal intervals
along the rim.

\begin{enumerate}[label = (\alph*)]

    \item A grasshopper sits at the 0. It can jump 5 units in either direction
(clockwise or counterclockwise) any number of times. What is the
set of numbers that the grasshopper can reach with a sequence of
jumps?

\framebox{\begin{minipage}{\dimexpr\linewidth-2\fboxrule-2\fboxsep}
        
    The set of numbers the grasshopper can reach is $\{0,1,\ldots,56\}$.    

        \begin{claim*}
            For each $n \in \Z$, there is some $k \in \Z$ for which $5k \equiv n \mod 57$.
        \end{claim*}

    \begin{proof}
        Since $\gcd(5,57)=1$, we know $\bar{5}$ is a unit in $\Z_{57}$. So there must be some $m\in \Z$ for which $5m \equiv 1 \mod 57$. Thus for any $n \in \Z$, we have $5m \cdot n \equiv 1 \cdot n$. But since $\Z$ is a ring, $mn = k$ for some $k \in \Z$ and $1 \cdot n = n$, so $5k \equiv n \mod 57$, as claimed.  
    \end{proof}

    The positions on the dial combined with the grasshopper's ability to jump is isomorphic to $\Z_{57}$. Consequently, the above result implies that the grasshopper can get to any position on the dial.
\end{minipage}}


    \item Same question, but instead the grasshopper can only jump 3 units
at a time.

\framebox{\begin{minipage}{\dimexpr\linewidth-2\fboxrule-2\fboxsep}
   
    Let $S=\{0,1,\ldots, 56\}$. The set of numbers the grasshopper can reach is given by $\{x \in S \mid x \equiv 0 \mod 3 \}$

    \begin{claim*}For each $m \in \Z$, there is some $n \in \Z$ for which $3m \equiv 3n \mod 57$.
    \end{claim*}

    \begin{proof}
        Let $m \in \Z$. Suppose otherwise that there is no $n \in \Z$ such that $3m \equiv 3n \mod 57$. Then either $3m \equiv 3n + 1 \mod 57$ or $3m \equiv 3n+2 \mod 57$. If $3m \equiv 3n+1 \mod 57$ for some $n \in \Z$, then $3(m-n) \equiv 1 \mod 7$, which implies $3$ is a unit. But this is impossible since $\gcd(3,57)=3\neq 1$. Now assume $3m \equiv 3n+2 \mod 57$. Then $3(m-n) \equiv 2 \mod 57$. Since $\gcd(2,57)=1$, we know there must be some integer $k$ satisfying $2k \equiv 1 \mod 57$. But then by transivitiy of the congruence relation, $3(m-n)k \equiv 1$, another contradiction.
    \end{proof}

\end{minipage}}


\end{enumerate}


\item Silverman 2.1

    \begin{enumerate}[label=(\alph*)]
        \item We showed that in any primitive Pythagorean triple $(a,b,c)$, either $a$ or $b$ is even. Use the same sort of argument to show that either $a$ or $b$ must be a multiple of $3$.


\framebox{\begin{minipage}{\dimexpr\linewidth-2\fboxrule-2\fboxsep}
    \begin{lemma*}There exists no integer whose square is congruent to $2 \mod 3$.
    \end{lemma*} 

    \begin{proof}
        Let $n \in \Z$. Then $n \equiv x \mod 3$ for exactly one $x \in \{0,1,2\}$. Suppose $n \equiv 0 \mod 3$. Then $n = 3k$ for some $k \in \Z$. So $n^2 = 9k^2= 3(3k^2)$. Thus $n^2 \equiv 0 \mod 3$.
        Suppose $n \equiv 1 \mod 3$. Then $n = 3k +1$ for some $k \in \Z$. So $n^2 = (3k+1)^2 = 9k^2+6k+1 = 3(3k^2+2k)+1$. Hence, $n^2 \equiv 1 \mod 3$.
        Suppose $n \equiv 2 \mod 3$. Then $n = 3k +2$ for some $k \in \Z$. So $n^2 = (3k+2)^2 = 9k^2+6k+4 = 3(3k^2+2k+1)+1$. Hence, $n^2 \equiv 1 \mod 2$.
    \end{proof}

    \begin{claim*}Let $(a,b,c)$ be a primitive Pythatgorean triple. Then either $a$ or $b$ is a multiple of $3$.
    \end{claim*}

    \begin{proof}Suppose otherwise that $3$ divides neither $a$ nor $b$. Then $a^2 \equiv b^2 \equiv 1 \mod 3$ since $x^2 \not\equiv 2 \mod 3$ for any $x \in \Z$ by the above lemma. So $a^2=3m+1$ and $b^2=3n+1$ for some $m,n \in \Z$. Since $\Z$ is a commutative ring, we have 
        \[c^2 = 3m+1+3n+1=3m+3n+2=3(m+n)+2.\] 
        But this implies $c^2 \equiv 2 \mod 3$, a contradiction.
    \end{proof}
\end{minipage}}

    
\item By examining the above list of primitive Pythagorean triples, make a guess about when $a$, $b$, or $c$ is a multiple of $5$. Try to show that your guess is correct.


\framebox{\begin{minipage}{\dimexpr\linewidth-2\fboxrule-2\fboxsep}

    \begin{lemma*}The square of any integer is congruent to $0$, $1$, or $4 \mod 5$.
    \end{lemma*}

    \begin{proof}
Let $n \in \Z$. Then $n \equiv x \mod 5$ for exactly one $x \in \set{0,\ldots,4}$. If $n \equiv 0 \mod 5$, then $n^2 = 0 \mod 5$. 
If $n \equiv 0 \mod 5$, then $n = 5m$ for some $m \in Z$. So $n^2 = 5k$ with $k = 5m^2$.
If $n \equiv 1 \mod 5$, then $n = 5m+1$ for some $m \in Z$. So $n^2 = 5k + 1$ with $k = 5m^2 + 2m$.
If $n \equiv 2 \mod 5$, then $n = 5m+2$ for some $m \in \Z$. So $n^2 = 5k + 4$ with $k = 5m^2 + 4m$.
If $n \equiv 3 \mod 5$, then $n = 5m+3$ for some $m \in \Z$. So $n^2 = 5k + 4$ with $k = 5m^2 + 6m + 1$.
If $n \equiv 4 \mod 5$, then $n = 5m+4$ for some $m \in \Z$. So $n^2 = 5k + 1$ with $k = 5m^2 + 8m + 3$.
    \end{proof}

    \begin{claim*}Let $(a,b,c)$ be a Primitive Pythagorean Triple. Then exactly one of $a$, $b$ or $c$ is congruent to $0 \mod 5$.
    \end{claim*}

    \begin{proof}
        Let $(a,b,c)$ be a primitive Pythagorean triple. Then $a$ and $b$ can not both be congruent to $0 \mod 5$, as then $c \equiv 0 \mod 5$ which contradicts primivity of $(a,b,c)$. Suppose neither $a$ nor $b$ is congruent $0 \mod 5$.
If both $a^2$ and $b^2$ are congruent to $1 \mod 5$, then $c^2$, and thus $c$, is congruent to $0 \mod 5$, which contradicts the above lemma. Similarly, it cannot be that both $a^2$ and $b^2$ are congruent to $4 \mod 5$, otherwise $c^2 = 3 \mod 5$, another contradiction. So if neither $a$ nor $b$
is congruent to $0 \mod 5$, then one of $a^2$ and $b^2$ must be congruent to $1 \mod 5$, and the other congruent to $4 \mod 5$.
Thus $c^2 \equiv 0 \mod 5$ which implies $c \equiv 0 \mod 5$. Now suppose neither $b$ nor $c$ is congruent to $0 \mod 5$.
We show $a \equiv 0 \mod 5$. Note that $a^2 = c^2 -b^2$. If $c^2 = 1 \mod 5$ and $b^2 \equiv 1 \mod 5$, then $a^2 = 0 \mod 5$, and thus
$a \equiv 0 \mod 5$. If both $c^2$ and $b^2$ are congruent to $4 \mod 5$, then $a^2 \equiv 0 \mod 5$. If $c^4 \equiv 4 \mod 5$ and
$b^2 \equiv 1 \mod 5$, then $a^2 = 3 \mod 5$, a contradiction.
If $c^2 = 1 \mod 5$ and $b^2 \equiv 4 \mod 5$, then $a^2 = 2 \mod 5$ and thus $a \equiv 2 \mod 5$, which is impossible. Therefore if neither $b$ nor $c$ is congruent to $0 \mod 5$, then $a$ must be. The result follows.
    \end{proof}
\end{minipage}}


    \end{enumerate}



\item Silverman 2.5

    In Chapter $1$ we saw that the $n^{\textnormal{th}}$ triangular $T_n$ is given by the formula
    \[T_n = 1 + 2 + 3 + \cdots + n = \frac{n(n+1)}{2}.\]
    The first few triangular numbers are $1$, $3$, $6$, and $10$. In the list of the first few Pythagorean triples $(a,b,c)$, we find $(3,4,5)$, $(5,12,13)$, $(7,24,25)$, and $(9,40,41)$. Notice that in each case, the value of $b$ is four times a triangular number.


\begin{enumerate}[label = (\alph*)]
    \item Find a primitive Pythagorean triple $(a,b,c)$ with $b=4T_5$. Do the same for $b=4T_6$ and for $b=4T_7$.

    
\framebox{\begin{minipage}{\dimexpr\linewidth-2\fboxrule-2\fboxsep}
    By applying the result from part (b) of this exercise, we get the following as the required Primitive Pythagorean Triples:\\
        $(a,4T_5,c)=(11,60,61)$\\
        $(a,4T_6,c)=(13,84,85)$\\
        $(a,4T_7,c)=(15,112,113)$
\end{minipage}}
    
    \item Do you think that for every triangular number $T_n$, there is a primitive Pythagorean triple $(a,b,c)$ with $b=4T_n$? If you believe that this is true then prove it. Otherwise, find some triangular number for which it is not true.


\framebox{\begin{minipage}{\dimexpr\linewidth-2\fboxrule-2\fboxsep}
    Yes.

    \begin{claim*}
        The triple $(a,b,c)=\left(2n+1,4T_n,2n^2+2n+1\right)$ yields a Primitive Pythagorean Triple for all $n \in \N_{>0}$.
    \end{claim*}
    
    \begin{proof}
    Let $T_n$ be the $nth$ triangular number, where $n \in \N_{>0}$. Then $4T_n=2n(n+1)$ since $T_n=\frac{n(n+1)}{2}$. Set $s = 2n+1$ and $t=1$. Then $s$ and $t$ are odd integers satisfying $s>t \geq 1$. Also, $s$ and $t$ must be relatively prime; otherwise, their common factor would divide both 
    $n+1 = \frac{s+t}{2}$ and $n = \frac{s-t}{2}$, contradicting the fact that $\gcd(n,n+1)=1$. Now observe that 
    $st=(2n+1)\cdot1 = 2n+1$, $4T_n=2n(n+1)=\frac{s^2-t^2}{2}$, and $(n+1)^2+1^2= \frac{s^2+t^2}{2}$. 
    Therefore, by Theorem 1, $\left(2n+1,4T_n,2n^2+2n+1\right)$ is a Primitive Pythagorean Triple. 
\end{proof}
\end{minipage}}

\end{enumerate}


\item Silverman 3.2 
    
\begin{enumerate}[label = (\alph*)]
    \item Use the lines through the point $(1,1)$ to describe all the points on the circle $x^2+y^2=2$ whose coordinates are rational numbers.

\framebox{\begin{minipage}{\dimexpr\linewidth-2\fboxrule-2\fboxsep}
Let $\ell$ be the line passing through point $(1,1)$. Then the equation for $\ell$ is given by $y-1 = m(x-1)$ which implies
$y= mx-m+1$. Now observe that
    \begin{align*}
        x^2+y^2&=2
        \\  x^2+\left(mx-m+1 \right)^2 &= 2
        \\ x^2+m^2x^2 +m^2+1 - 2m^2x-2m+2mx &=2
        \\(m^2+1)x^2-2(m^2-m)x+(m^2-2m-1)&=0
    \end{align*}
    Then dividing $(m^2+1)x^2-2(m^2-m)x+(m^2-2m-1)$ by $x-1$, we obtain
    \[(m^2+1)x -(m^2-2m-1)\]
So
    \[x = \frac{m^2-2m-1}{m^2+1}\]
Thus
       \[ y = m\left( \tfrac{m^2-2m-1}{m^2+1}\right) - m +1 = \tfrac{-m^2-2m+1}{m^2+1}\]
    Hence $(x,y) = \left( (m^2+1)x -(m^2-2m-1),  \tfrac{-m^2-2m+1}{m^2+1}\right)$.
\end{minipage}}

        
        \begin{comment}
    \item What goes wrong if you try to apply the same procedure to find all the points on the circle $x^2+y^2=3$ with rational coordinates?
        \end{comment}


\end{enumerate}


\item Silverman 5.1 

Use the Euclidean algorithm to compute each of the following gcd's.


\begin{enumerate}[label = (\alph*)]
    \item $\gcd(12345,67890)$


\framebox{\begin{minipage}{\dimexpr\linewidth-2\fboxrule-2\fboxsep}
        \begin{align*}
            \gcd(12345,67890) &= \gcd(12345, 67890 - 5(12345))
            \\&= \gcd(12345, 6165)
            \\&= \gcd(12345-2(6165), 6165)
            \\&= \gcd(15,6165)
            \\&= \gcd(15, 6165 - 411(15))
            \\&= \gcd(15,0)
            \\&= 15
        \end{align*}
\end{minipage}}

    \item $\gcd(54321,9876)$

        
\framebox{\begin{minipage}{\dimexpr\linewidth-2\fboxrule-2\fboxsep}
        \begin{align*}
            \gcd(54321,9876) &= \gcd(54321 - 5(9875), 9876)
            \\&=\gcd(4941,9876)
            \\&=\gcd(4941,9875-1(4941))
            \\&=\gcd(4941,4935)
            \\&=\gcd(4941-1(4935),4935)
            \\&=\gcd(6,4935)
            \\&=\gcd(6,4935-822(6))
            \\&=\gcd(6,3)
            \\&=\gcd(6-2(3),3)
            \\&=\gcd(0,3)
            \\&= 3
        \end{align*}
\end{minipage}}

\end{enumerate}

\item Silverman 5.6 The proof should be very short!

    Write a program to implement the $3n+1$ algorithm described in the previous exercise. The user will input $n$ and your program should return the length $L(n)$ and the previous terminating value $T(n)$ of the $3n+1$ algorithm. Use your program to create a table giving the length and terminating value for all starting values $1 \leq n \leq 100$.


\begin{minipage}{\textwidth}
    \begin{lstlisting}    
def g(n):
    """Compute Length of Termination and Terminating value"""
    A, i = [], 0
    while n not in A:
        A, i = A + [n], i + 1
        if n % 2 == 0:
            n = n // 2
        else:
            n = (3 * n) + 1
    return i, A[i-1]


def f(k):
    """Print table for Length of Termation and Terminating values"""
    for n in range(1, k+1):
        print(n, g(n))
\end{lstlisting}


\end{minipage}


\section*{appendix}
$3n+1$ algorithm output
\begin{lstlisting}
1 (3, 2)
2 (3, 4)
3 (8, 1)
4 (3, 1)
5 (6, 1)
6 (9, 1)
7 (17, 1)
8 (4, 1)
9 (20, 1)
10 (7, 1)
11 (15, 1)
12 (10, 1)
13 (10, 1)
14 (18, 1)
15 (18, 1)
16 (5, 1)
17 (13, 1)
18 (21, 1)
19 (21, 1)
20 (8, 1)
21 (8, 1)
22 (16, 1)
23 (16, 1)
24 (11, 1)
25 (24, 1)
26 (11, 1)
27 (112, 1)
28 (19, 1)
29 (19, 1)
30 (19, 1)
31 (107, 1)
32 (6, 1)
33 (27, 1)
34 (14, 1)
35 (14, 1)
36 (22, 1)
37 (22, 1)
38 (22, 1)
39 (35, 1)
40 (9, 1)
41 (110, 1)
42 (9, 1)
43 (30, 1)
44 (17, 1)
45 (17, 1)
46 (17, 1)
47 (105, 1)
48 (12, 1)
49 (25, 1)
50 (25, 1)
51 (25, 1)
52 (12, 1)
53 (12, 1)
54 (113, 1)
55 (113, 1)
56 (20, 1)
57 (33, 1)
58 (20, 1)
59 (33, 1)
60 (20, 1)
61 (20, 1)
62 (108, 1)
63 (108, 1)
64 (7, 1)
65 (28, 1)
66 (28, 1)
67 (28, 1)
68 (15, 1)
69 (15, 1)
70 (15, 1)
71 (103, 1)
72 (23, 1)
73 (116, 1)
74 (23, 1)
75 (15, 1)
76 (23, 1)
77 (23, 1)
78 (36, 1)
79 (36, 1)
80 (10, 1)
81 (23, 1)
82 (111, 1)
83 (111, 1)
84 (10, 1)
85 (10, 1)
86 (31, 1)
87 (31, 1)
88 (18, 1)
89 (31, 1)
90 (18, 1)
91 (93, 1)
92 (18, 1)
93 (18, 1)
94 (106, 1)
95 (106, 1)
96 (13, 1)
97 (119, 1)
98 (26, 1)
99 (26, 1)
100 (26, 1)
\end{lstlisting}

\end{exercises}
\end{document}
