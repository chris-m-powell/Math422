\documentclass[10pt,twoside]{article}

% \usepackage[utf8]{inputenc}
\usepackage{geometry}
\usepackage{fancyhdr}
\usepackage{amsmath,amssymb,amsthm}
\usepackage{mathrsfs}
\usepackage{enumitem}
\usepackage{verbatim}
\usepackage{parskip}
\usepackage{xcolor}
\usepackage{array} 
\usepackage{url} 
\usepackage{float}
\usepackage{braket}
\usepackage{listings}
\usepackage{inconsolata}
\usepackage{mystyle}
\usepackage{lastpage}
\usepackage{appendix}
\usepackage{fontspec}
\usepackage[euler-digits,euler-hat-accent]{eulervm}

\raggedbottom
\setmainfont{URW Palladio L}
% \newfontfamily\headingfont[Path = /usr/share/fonts/opentype/bebas-neue/]{BebasNeue Regular.otf}
% \setmainfont[Path = /usr/share/fonts/opentype/tex-gyre-pagella/]{texgyrepagella-regular.otf}
\begin{document}
\title{Homework 4}
\author{Chris Powell}
\date{}
\maketitle
\thispagestyle{firststyle}


\begin{enumerate}[itemsep=1em,label=\Alph*.]

\item 
    Let $d \mid a$, $d \mid b$. Show that
    \[\gcd(a, b) = d \gcd\left(\frac{a}{d},\frac{b}{d}\right).\]


\framebox{\begin{minipage}{\dimexpr\linewidth-2\fboxrule-2\fboxsep}
    \begin{proof}
            As $d \mid a,b$, we know $\frac{a}{d}, \frac{b}{d}\in \Z$. By the Linear Equation Theorem, we know that
            $\gcd\left(a, b\right) = ax+by$ for some $x,y \in \Z$. Likewise, there exists $s,t \in \Z$ such that $\gcd\left(\frac{a}{d},\frac{b}{d}\right)=\frac{a}{d}s+\frac{a}{d}t$. Note that $\gcd(a,b)$ and $\gcd \left( \frac{a}{d},\frac{b}{d} \right)$ are the least postive values of $ax+by$ and $\frac{a}{d}s+\frac{b}{d}t$, respectively. Thus 
    \[d \gcd\left(\frac{a}{d},\frac{b}{d}\right) = d \left(\frac{a}{d}s+ \frac{b}{d}t\right) = as + bt \geq ax+by = \gcd(a,b).\]
On the other hand,
            \[d\gcd\left(\frac{a}{d},\frac{b}{d}\right)=d\left(\frac{a}{d}s+ \frac{b}{d}t \right) \leq d\left(\frac{a}{d}x+\frac{b}{d}y\right) = ax+by= \gcd(a,b).\]
Hence $\gcd(a,b) = \gcd\left(\frac{a}{d},\frac{b}{d}\right)$.
        \end{proof}
\end{minipage}}






\item Silverman 7.3. Let $s$ and $t$ be odd integers with $s>t \geq 1$ and $\gcd(s,t)=1$. Prove that the three numbers
    \[st, \quad \frac{s^2-t^2}{2}, \quad \textnormal{and} \quad \frac{s^2+t^2}{2}\]
are pairwise relatively prime.


\framebox{\begin{minipage}{\dimexpr\linewidth-2\fboxrule-2\fboxsep}
        \begin{proof}
            Since $\left(st, \frac{s^2-t^2}{2}, \frac{s^2+t^2}{2} \right)$ is a primitive Pythagorean triple, it suffices to show that $g=\gcd\left(st, \frac{s^2-t^2}{2}\right)=1$, as any pairwise common divisor must be a common divisor of the other pairs. Suppose otherwise that $g >1$. Then by the Fundamental Theorem of Arithmetic, there exists a prime factor $p$ of $g$. Thus $p \mid st, \frac{s^2-t^2}{2}$ by transitivity of the divisibility relation. Therefore either $p \mid s$ or $p \mid t$ since $p$ is prime. Suppose $p \mid s$. Since $p \mid \frac{s^2-t^2}{2}$, we have that $\frac{s^2-t^2}{2}=pk$ for some $k \in \Z$. Thus
            \[2pk = s^2-t^2 = (s+t)(s-t).\]
            So $p \mid (s+t)(s-t)$ which implies that either $p \mid s+t$ or $p \mid s-t$. Either way, this implies $p \mid t$. So $p \mid s,t$. But $\gcd(s,t)=1$, a contradiction. A similar argument holds if $p \mid t$.
        \end{proof}
\end{minipage}}






\item Recall that for $n \in \N$, $n!$ means $n · (n - 1) \cdots 2 \cdot 1$. How many $0s$ does $100!$
end in?


\framebox{\begin{minipage}{\dimexpr\linewidth-2\fboxrule-2\fboxsep}
Note that $100! = 100 \cdot 99 \cdots 2 \cdot 1$. A trailing $0$ is formed from the product of a multiple of some power of $5$ and a multiple of some power of $2$. So we add the powers of $5$ which divide $100$: 
        \[\sum_{k : 5^k \mid 100} \frac{100}{5^k} = \frac{100}{5^1} +\frac{100}{5^2}= \frac{100}{5} +\frac{100}{25} = 20+4=24.\]
    Since there are clearly more multiples of $2$ than $5$ in $\set{1,\ldots,100}$, we conlude that there are $24$ trailing zeros in $100!$.
\end{minipage}}




\item Let $n \in \N$ have prime factorization $p_1^{e_1}p_2^{e_2}\cdots p_r^{e_r}$, where the $p_i$ are distinct primes and the $e_i \geq 1$. Show that $n$ is a perfect square if, and only if, $2 \mid e_i$ for all $i$.


\framebox{\begin{minipage}{\dimexpr\linewidth-2\fboxrule-2\fboxsep}
    \begin{proof}
        Suppose $n$ is a perfect square. Then $n=m^2$ for some $m \in \Z$. By the Fundamental Theorem of Arithmetic, $m$ has a unique prime factorization $m = \prod_{i=1}^s q^{d_i}_i$, where each $d_i$ is distinct. Since this factorization is unique (up to rearrangment),  we know that $r=s$ and that there is some permutation $\sigma$ such that $p_i=\sigma(q_i)$ for all $i\in \set{1,\ldots,r}$. Hence 
            \[n = \left( \prod_{i=1}^s q_i^{d_i} \right)^2= \prod_{i=1}^s \left(q_i^{d_i}\right)^2=\prod_{i=1}^s q_i^{2d_i}.\]
        By uniqueness, $e_i=2d_i$ for all $i$. Thus $2 \mid e_i$ for all $i$.
        Conversely, if $2 \mid e_i$ for all $i$, then
        \[n = \prod_{i=1}^r p_i^{2d_i} = \prod_{i=1}^r \left(p_i^{e_i}\right)^2 = \left(\prod_{i=1}^r p_i^{e_i} \right)^2.\]
        Since $\Z$ is closed under multiplication, $n = m^2$ for some $m \in \Z$.
        \end{proof}
\end{minipage}}






\item Silverman 8.5 Find all incongruent solutions to each of the following linear congruences.


    \begin{enumerate}[itemsep=1em, label=(\alph*)]
        \item $8x \equiv 6 \mod{14}$

\framebox{\begin{minipage}{\dimexpr\linewidth-2\fboxrule-2\fboxsep}
            First, we apply the Euclidean algorithm to evaluate $g=\gcd(8,14)$:
            \begin{align*}
                14  &= 1(8)+6 \quad (q_1,r_1)=(1,6)
                \\8  &= 1(6)+2 \quad (q_2,r_2)=(1,2)
                \\6  &= 3(2)+0 \quad (q_3,r_3)=(3,0). 
            \end{align*}
            Since $r_3=0$, it follows that $g=r_2=2$. Since $2 \mid 14$, the Linear Congruence Theorem implies
            that $8x \equiv 6 \mod{14}$ has exactly $2$ incongruent solutions. By the Linear Equation Theorem, we can find $u,v \in \Z$ which satisfy 
            \[8u+14v=2. \quad (*)\]
            Write $a=14$ and $b=8$. Then
            \begin{align*}
                r_1 &=a - q_1b
                \\  &=a - (1)b
            \\r_2   &=b - q_2r_1
                  \\&=b - (1)(a-(1)b)
                  \\&=(-1)a+2b
            \end{align*}
            Thus $(u,v)=(2,-1)$ is a solution to $(*)$. So 
            $3\cdot (u,v) = (6,-3)$ is a solution to
            \[8u+14v=6. \quad (**)\]
            Therefore, by the Linear Congruence Theorem, $x=6\frac{u}{g}=6$ is a solution to $8x \equiv 6 \mod{14}$, and the set of all incongruent solutions is given by $x \equiv 6 +7k$ for all $k \in \set{0,1}$. Explicitly, this yields $x \equiv 6 \mod{14}$ and $x \equiv 13 \mod{14}$.
    \end{minipage}}
        
        \item $66x \equiv 100 \mod{121}$


\framebox{\begin{minipage}{\dimexpr\linewidth-2\fboxrule-2\fboxsep}
            First, we apply the Euclidean algorithm to evaluate $\gcd(66,121)$:
            \begin{align*}
                121  &= 1(66)+55 \quad (q_1,r_1)=(1,55)
                \\66 &= 1(55)+11 \quad (q_2,r_2)=(1,11)
                \\55  &= 5(11)+0 \quad (q_3,r_3)=(5,0).
            \end{align*}
            Since $r_3=0$, we know $\gcd(66,121)=r_2=11$. But since $11 \nmid 100$, the Linear Congruence Theorem implies that $66x \equiv 100 \mod{121}$ has no solutions, i.e.,
            \[\set{x \in \Z \mid 66x \equiv 100 \mod{121}} = \emptyset.\]
\end{minipage}}

        \item $21x \equiv 14 \mod{91}$


\framebox{\begin{minipage}{\dimexpr\linewidth-2\fboxrule-2\fboxsep}
            We first apply the Euclidean algorithm to evaluate $g=\gcd(21,91)$:
            \begin{align*}
                91   &= 4(21)+7 \quad (q_1,r_1)=(4,7)
                \\21  &= 3(7)+0 \quad (q_2,r_2)=(1,2)
            \end{align*}
            Since $r_2=0$, we know $g=r_1=7$. Since $7 \mid 14$, the Linear Congruence Theorem implies
            that $21x \equiv 14 \mod{91}$ has exactly $7$ incongruent solutions. By the Linear Equation Theorem, we know there exists $u,v \in \Z$ satisfying 
            \[21u+91v=7. \quad (*)\]
            Write $a=91$ and $b=21$. Then
            \begin{align*}
                r_1 &=a - q_1b
                \\  &=a - (4)b
            \end{align*}
            Thus $(u,v)=(-4,1)$ is a solution to $(*)$. So 
            $2\cdot (u,v) = (-8,2)$ is a solution to
            \[21u+91v=14. \quad (**)\]
            Therefore, by the Linear Congruence Theorem, $x=14\frac{u}{g}=-8$ is a solution to $21x \equiv 14 \mod{91}$, and the set of all incongruent solutions is given by $x \equiv -8 +(13)k$ for all $k \in \set{0,\ldots,6}$. Explicitly, this yields
            incongruent solutions $[5], [18], [31], [44], [57], [70]$, and $[83]$.
    \end{minipage}}

    \end{enumerate}









\item Silverman 8.6 Determine the number of incongruent solutions for each of the following congruences. You do not need to write down the actual solutions.


    \begin{enumerate}[itemsep=1em, label=(\alph*)]
        \item $72x \equiv 47 \mod{200}$


\framebox{\begin{minipage}{\dimexpr\linewidth-2\fboxrule-2\fboxsep}
            We first apply the Euclidean algorithm to evaluate $g=\gcd(72,200)$:
            \begin{align*}
                200   &= 2(72)+56 \quad (q_1,r_1)=(2,56)
                \\72  &= 1(56)+16 \quad (q_2,r_2)=(1,16)
                \\56  &= 3(16)+8 \quad (q_3,r_3)=(3,8)
                \\16  &= 2(8)+0 \quad (q_4,r_4)=(2,8)
            \end{align*}
            Since $r_4=0$, we have that $\gcd(72,200)=r_3=8$. But since $8 \nmid 47$, the Linear Congruence Theorem implies that $72x \equiv 47 \mod{200}$ has no soluitions, i.e.,
            \[\set{x \in \Z \mid 72x \equiv 47 \mod{200}}=\emptyset.\]
    \end{minipage}}

        \item $4183x \equiv 5781 \mod{15087}$

\framebox{\begin{minipage}{\dimexpr\linewidth-2\fboxrule-2\fboxsep}
            Since $\gcd(4183,15087) = 47 \mid 5781$, the Linear Congruence Theorem implies that $4183x \equiv 5781 \mod{15087}$ has $47$ incongruent solutions.
\end{minipage}}
        
        \item $1537x \equiv 2863 \mod{6731}$


\framebox{\begin{minipage}{\dimexpr\linewidth-2\fboxrule-2\fboxsep}
            Since $\gcd(1537,6731) = 53 \nmid 2863$, the Linear Congruence Theorem implies that $1537x \equiv 2863 \mod{6731}$ has no solution, i.e.,
            \[\set{x \in \Z \mid 1537x \equiv 2863 \mod{6731}}=\emptyset.\]
\end{minipage}}

    \end{enumerate}








\item Silverman 8.7 Write a program that solves the congruence
    \[ax \equiv c \mod{m}.\]

\begin{lstlisting}
def xgcd(a, b):
    """Return (g, x, y) such that a*x + b*y = g = gcd(a, b)"""
    if b == 0:
        return a, 1, 0
    x, g, v, w = 1, a, 0, b
    while w != 0:
        x, g, v, w = v, w, x - (g // w) * v, g % w
    x = x % (b // g)
    y = (g - (a * x)) // b
    return g, x, y


def cong(a, c, m):
    """Return {x in Z | ax = c (mod m)}"""
    g = xgcd(a, m)[0]
    S = []
    if c % g != 0:
        return S  # solution-set S is empty
    else:
        x = (c * xgcd(a, m)[1]) // g
        for i in range(0, g):
            S = S + [(x + i * (m // g)) % m]
    return S  # return g incongruent solutions
\end{lstlisting}


\end{enumerate}



\end{document}
