\documentclass[10pt,twoside]{article}

% \usepackage[utf8]{inputenc}
\usepackage{geometry}
\usepackage{fancyhdr}
\usepackage{amsmath,amssymb,amsthm}
\usepackage{mathrsfs}
\usepackage{enumitem}
\usepackage{verbatim}
\usepackage{parskip}
\usepackage{xcolor}
\usepackage{array} 
\usepackage{url} 
\usepackage{float}
\usepackage{braket}
\usepackage{listings}
\usepackage{inconsolata}
\usepackage{mystyle}
\usepackage{lastpage}
\usepackage{fontspec}
\usepackage[euler-digits,euler-hat-accent]{eulervm}


\raggedbottom
\setmainfont{URW Palladio L}
% \newfontfamily\headingfont[Path = /usr/share/fonts/opentype/bebas-neue/]{BebasNeue Regular.otf}
% \setmainfont[Path = /usr/share/fonts/opentype/tex-gyre-pagella/]{texgyrepagella-regular.otf}
\begin{document}
\title{\vspace{-2em}Homework 8\vspace{-1em}}
\author{Chris Powell}
\date{}
\maketitle
\thispagestyle{fancy}

\begin{enumerate}[itemsep=2em,label=\Alph*.]
    \item Silverman 16.1. Use the method of successive squaring to compute each of the following powers.
        \begin{enumerate}[label=(\alph*)]
            \item $5^{13} \mod{23}$
                
                \framebox{\begin{minipage}{\dimexpr\linewidth-2\fboxrule-2\fboxsep}
                    First, we find the binary expansion of the exponent 13:
                \begin{align*}
                    13 &= 1 \cdot 2^3 + 1 \cdot 2^2 + 0 \cdot 2^1 + 1 \cdot 2^0
                    \\&=  8 + 4 + 0 + 1.
                \end{align*}
                    Then we compute $5^k \mod{23}$ for each $k \in \set{2^0,\ldots,2^3}$:
                \begin{align*}
                    5^1 & \equiv 5 \mod{23}
                    \\5^2 &\equiv 25 \equiv 2 \mod{23}
                    \\5^4 &\equiv (5^2)^2 \equiv 2^2 \equiv 4 \mod{23}
                    \\5^8 &\equiv (5^4)^2 \equiv 4^2 \equiv 16 \mod{23}.
                \end{align*}
                Therefore, by Algorithm 16.1, $5^{13} \equiv 16^1 \cdot 4^1 \cdot 2^0  \cdot 5^1 \equiv 320 \equiv 21 \mod{23}$.
                \end{minipage}}

            \item $28^{749} \mod{1147}$

                \framebox{\begin{minipage}{\dimexpr\linewidth-2\fboxrule-2\fboxsep}
                    First, we find the binary expansion of the exponent $749$:
                \begin{align*}
                    749&= 1\cdot2^9 + 0\cdot 2^8 + 1\cdot 2^7 +1\cdot 2^6 + 1 \cdot 2^5 
                    \\&+ 1 \cdot 2^4 + 0 \cdot 2^3 + 1 \cdot 2^2 +0 \cdot 2^1 + 1 \cdot 2^0
                    \\&= 512 + 128 + 64 + 32 + 8 + 4 + 1.
                \end{align*}
                    Then we compute $28^k \mod{1147}$ for each $k \in \set{2^0,\ldots,2^9}$:
                \begin{align*}
                    28^1 &\equiv 28 \mod{1147}
                    \\28^2 &\equiv 784 \mod{1147}
                    \\28^4 &\equiv (28^2)^2 \equiv 784^2 \equiv 614656 \equiv 1011 \mod{1147}
                    \\28^8 &\equiv (28^4)^2 \equiv 1011^2 \equiv 1022121 \equiv 144 \mod{1147}
                    \\28^{16} &\equiv (28^8)^2 \equiv 144^2 \equiv 20736 \equiv 90 \mod{1147}
                    \\28^{32} &\equiv (28^{16})^2 \equiv 90^2 \equiv 8100 \equiv 71 \mod{1147}
                    \\28^{64} &\equiv (28^{32})^2 \equiv 71^2 \equiv 5041 \equiv 453 \mod{1147}
                    \\28^{128} & \equiv (28^{64})^2 \equiv 453^2 \equiv 205209 \equiv 1043 \mod{1147}
                    \\28^{256} &\equiv (28^{128})^2 \equiv 1043^2 \equiv 1087849 \equiv 493 \mod{1147}
                    \\28^{512} &\equiv (28^{256})^2 \equiv 493^2 \equiv 243049 \equiv 1032 \mod{1147}.
                \end{align*}
                Therefore, by Algorithm 16.1, 
                    \begin{align*}28^{749} \equiv& 1032^1 \cdot 493^0 \cdot 1043^1 \cdot 453^1 \cdot 71^1 
                        \\\cdot& 90^1 \cdot 144^0 \cdot 1011^1 \cdot 784^0 \cdot 28^1 
                        \\ \equiv& 289 \mod{1147}.
                    \end{align*}
                \end{minipage}}


        \end{enumerate}
    \item Silverman 16.2c. The method of successive squaring described in the text allows you to compute $a^k \mod{m}$ quite efficiently, but it does involve creating a table of powers of $a$ modulo $m$.
            \begin{enumerate}[label=(\alph*)]
                \setcounter{enumii}{2}
            \item Use your program to compute the following quantities:
                \begin{enumerate}[label=(\roman*)]
                    \item $2^{1000} \mod{2379}$

                    \begin{lstlisting}
def expmod(a, k, m):
    """compute a^k mod m"""
    b = 1
    while k:
        if k % 2 == 1:
            b = (b * a) % m
        a, k = (a ** 2) % m, k // 2
    return b
    \end{lstlisting}

                        \framebox{\begin{minipage}{\dimexpr\linewidth-2\fboxrule-2\fboxsep}
                        $2^{1000} \equiv 562 \mod{2379}$ 
                \end{minipage}}    
                    
                    \item $567^{1234} \mod{4321}$
                        
                \framebox{\begin{minipage}{\dimexpr\linewidth-2\fboxrule-2\fboxsep}
                        $567^{1234} \equiv 3214 \mod{4321}$
                \end{minipage}}    
                    \item $47^{258008} \mod{1315171}$

                \framebox{\begin{minipage}{\dimexpr\linewidth-2\fboxrule-2\fboxsep}
                        $47^{258008} \equiv 1296608 \mod{1315171}$
                \end{minipage}}
                \end{enumerate}
        \end{enumerate}
    
    
    \item Silverman 16.3.
        \begin{enumerate}[label=(\alph*)]
            \item Compute $7^{7386} \mod{7387}$ by the method of successive squaring. Is $7387$ prime?
                
                \framebox{\begin{minipage}{\dimexpr\linewidth-2\fboxrule-2\fboxsep}
                Since $7^{7386} \equiv 702 \not\equiv 1 \mod{7387}$, Fermat's Little Theorem implies that the modulus $7387$ is not prime.
                \end{minipage}}

            \item Compute $7^{7392} \mod{7393}$ by the method of successive squaring. Is $7393$ prime? 

                \framebox{\begin{minipage}{\dimexpr\linewidth-2\fboxrule-2\fboxsep}
                Since $7^{7392} \equiv 1 \mod{7393}$, Fermat's Little Theorem implies that the modulus $7393$ is prime.
                \end{minipage}}
        
        \end{enumerate}
    \item Silverman 16.4. Ignore the second paragraph. To generate random numbers, put import random at the top of your file, then call random.randint(a,b) to get a random number between a and b inclusive. Write a program to check if a number $n$ is composite or probably prime as follows. Choose $10$ randoms numbers $a_1,\ldots,a_{10}$ between $2$ and $n-1$ and compute $a_i^{n-1} \mod{n}$ for each $a_i$. If $a_i^{n-1} \not\equiv 1 \mod{n}$ for any $a_i$, return the message "n is composite." If $a_i^{n-1} \equiv 1 \mod{n}$ for all the $a_i$'s, return the message "n is probably prime"
   
   
        \begin{lstlisting}
import random

        
def expmod(a, k, m):
    """compute a^k mod m"""
    b = 1
    while k:
        if k % 2 == 1:
            b = (b * a) % m
        a, k = (a ** 2) % m, k // 2
    return b


def probablyprime(n):
    """Returns whether integer n is likely prime or composite"""
    A = []
    for i in range(10):
        A += [random.randint(2, n-1)]
    for i in A:
        if expmod(A[i], n-1, n) % n != 1:
            return str(n) + " is composite"
    return str(n) + " is prime"
        \end{lstlisting}                


    \item 
        \begin{enumerate}[label=\arabic*.]
            \item Show that if $\gcd(a,n)=1$ and $r \equiv s \mod{\varphi(n)}$, then $a^r \equiv a^s \mod{n}$.

                \framebox{\begin{minipage}{\dimexpr\linewidth-2\fboxrule-2\fboxsep}
                \begin{proof}
                    As $\gcd(a,n)=1$, Euler's formula implies $a^{\varphi(n)} \equiv 1 \mod{n}$. Since $r \equiv s \mod{\varphi(n)}$, we have $r = \varphi(n)k+s$ for some integer $k$. Take $r,s >1$. Then
                    \begin{align*}
                        a^r &\equiv a^{\varphi(n)k+s} \equiv \left(a^\varphi(n) \right)^k \cdot a^s \equiv 1^k \cdot a^s \equiv a^s \mod{n}.
                    \end{align*}
                \end{proof}
                \end{minipage}} 
            \item Show that if $\gcd(a,n)\neq1$, the above is not necessarily true.

                \framebox{\begin{minipage}{\dimexpr\linewidth-2\fboxrule-2\fboxsep}
                    Consider $(a,n,r,s)=(2,8,13,1)$. Then $\gcd(2,8)=2 \neq 1$, $1 \equiv 13 \mod{4}$, and $\varphi(8)=\varphi(2^3)=2^3-2^2=4$. But
                    $2^1 \not\equiv 2^{13} \mod{8}$.
                \end{minipage}}
        \end{enumerate}
    \item Silverman 17.2
        \begin{enumerate}[label=(\alph*)]
            \item Solve the congruence $x^{113} \equiv 347 \mod{463}$.
                
                \begin{comment}
                \begin{lstlisting}
def gcd(a, b):
    """Return gcd(a,b)"""
    while b:
        a, b = b, a % b
    return a


def xgcd(a, b):
    """Return (g, x, y) such that a*x + b*y = g = gcd(a, b)"""
    if b == 0:
        return a, 1, 0
    x, g, v, w = 1, a, 0, b
    while w != 0:
        x, g, v, w = v, w, x - (g // w) * v, g % w
    x = x % (b // g)
    return g, x, (g - (a * x)) // b


    t = 0
    for i in range(1, n):
        if gcd(i, n) == 1:
            t += 1
    return t

                
def expmod(a, k, m):
    """compute a^k mod m"""
    b = 1
    while k:
        if k % 2 == 1:
            b = (b * a) % m
        a, k = (a ** 2) % m, k // 2
    return b


def modroot(k, b, m):
    """return x such that x^k cong b mod m if gcd(b,m)=1, gcd(k, totient(m)"""
    if gcd(b, m) == 1:
        (g, u, v) = xgcd(k, totient(m))
        if g == 1:
            return expmod(b, u, m)
                \end{lstlisting}
        \end{comment}            
                \framebox{\begin{minipage}{\dimexpr\linewidth-2\fboxrule-2\fboxsep}
                    By the Euclidean algorithm, we know $\gcd(347,463)=1$. Also,
                    by the Linear Equation Theorem, we know there exists $u,v \in \Z_{>0}$ satisfying
                    \[113 u - \varphi(463) v = \gcd(113,462).\]
                    But since $463$ is prime, we know $\varphi(463)=463^1-463^0=462$. By applying the extended Euclidean algorithm, we find
                    \[\gcd(113,462)=1, \quad (u,v) = (323,79).\]
                    Now observe that
                    \begin{align*}
                        \left(347^{323}\right)^{113} &=347^{323 \cdot 113}
                        \\&=347^{1+\varphi(463)(79)} 
                        % \\&=347\cdot 347^{\varphi(463) \cdot 79}
                        \\&=347\cdot \left(347^{\varphi(463)}\right)^{79}
                        \\&\equiv347\cdot 1^{79} \justify{Euler's formula}
                        \\&\equiv 347 \mod{463}.
                    \end{align*}
                    Using successive squaring, we compute 
                    \[347^{113} \equiv 37 \mod{473}.\] 
                    Hence $x=37$ satisfies $x^{113} \equiv 347 \mod{463}$.
                \end{minipage}}
                \pagebreak
            \item Solve the congruence $x^{275} \equiv 139 \mod{588}$. 
                
                \framebox{\begin{minipage}{\dimexpr\linewidth-2\fboxrule-2\fboxsep}
                    By the Euclidean algorithm, we know $\gcd(139,588)=1$. Also,
                    by the Linear Equation Theorem, we know there exists $u,v \in \Z_{>0}$ satisfying
                    \[275 u - \varphi(588) v = \gcd(275,588).\]
                    But
                    \begin{align*}
                        \varphi(588)&=\varphi(2^2)\varphi(3^1)\varphi(7^2)
                        \\&=(2^2-2^1)(3^1-3^0)(7^2-7^1)
                        \\&=168. 
                    \end{align*}
                    By applying the extended Euclidean algorithm, we find
                    \[\gcd(275,168)=1, \quad (u,v) = (11,18).\]
                     Now observe that
                    \begin{align*}
                        \left(139^{11}\right)^{275} &=139^{11 \cdot 275}
                        \\&=139^{1+\varphi(588)(18)} 
                        % \\&=347\cdot 347^{\varphi(463) \cdot 79}
                        \\&=139\cdot \left(139^{\varphi(588)}\right)^{18}
                        \\&\equiv 139\cdot 1^{18} \justify{Euler's formula}
                        \\&\equiv 139 \mod{463}.
                    \end{align*}
                    Using successive squaring, we compute 
                    \[139^{11} \equiv 559 \mod{588}.\] 
                    Thus $x=559$ satisfies $x^{275} \equiv 139 \mod{588}$.
                \end{minipage}} 
                
                \end{enumerate}
    \item Silverman 17.4
            Our method for solving $x^k \equiv b \mod{m}$ is first to find integers $u$ and $v$ satisfying $ku-\varphi(m)v=1$, and then the solution is $x \equiv b^u \mod{m}$. However, we only showed that this works provided that $\gcd(b,m)=1$, since we used Euler's formula $b^{\varphi(m)} \equiv 1 \mod{m}$.
        \begin{enumerate}[label=(\alph*)]
            
            \item If $m$ is a product of distinct primes, show that $x \equiv b^u \mod{m}$ is always a solution to $x^k \equiv b \mod{m}$, even if $\gcd(b,m)>1$.
                
                \framebox{\begin{minipage}{\dimexpr\linewidth-2\fboxrule-2\fboxsep}
                \begin{proof}
                    Assume $m$ has prime factorization $m= \prod_{i=1}^r p_i$, where each $p_i$ is distinct. Then 
                    \begin{align*}
                        \varphi(m)&=\varphi \left(\prod_{i=1}^r p_i \right) =\prod_{i=1}^r \varphi(p_i) =\prod_{i=1}^r (p_i-1)
                    \end{align*}
                    which implies $p_i - 1 \mid \varphi(m)$ for all $i$. Thus for each $i$, $\varphi(m)=(p_i-1)k$ for some $k \in \Z$. We know
                    $\prod_{i=1}^r p_i \mid (b^u)^k - b$ as $(b^u)^k\equiv b \mod{m}$. We claim $p_i \mid (b^u)^k-b$ for all $i$. 
                    
                    Now either $p_i \mid b$ for all $i$ or there is some $p_i$ for which $p_i \nmid b$. If $p_i \mid b$, then $p_i \mid (b^u)^k -b$. Suppose there is some $p_i$ which does not divide $b$.
                    We know there exists $u,v \in \Z$ satisfying $ku-\varphi(m)=1$. But this implies 
                    \[ku=1-\varphi(m)v=1+\left((p_j-1)k\right)v.\]
Thus,
                    \begin{align*}
                        (b^u)^k &= b^{uk}
                                \\&=b^{1+(p_j-1)kv}
                                \\&=b\cdot b^{(p_i-1)kv}
                                \\&=b\cdot \left(b^{p_i-1}\right)^{kv}
                                \\&\equiv b\cdot 1^{kv} \mod{p_i} \justify{Fermat's Little Theorem}
                                \\&\equiv b \mod{p_i}.
                    \end{align*}
                    So $p_i \mid (b^u)^k-b$. Thus every $p_i$ divides $(b^u)^k-b$, as claimed. The result follows.
                \end{proof}
                \end{minipage}}

            \item Show that our method does not work for the congruence $x^5 \equiv 6 \mod{9}$.

        \framebox{\begin{minipage}{\dimexpr\linewidth-2\fboxrule-2\fboxsep}
            Note that $\gcd(6,9)=3\neq1$. By applying the extended Euclidean algorithm, we find that $(u,v)=(5,4)$ satisfies $5u-6v=1$.
            Now see that $6^5 \equiv 0 \mod{9}$, yet $0^5 \equiv 6 \mod{9}$. Hence, the given congruence admits no solutions.
        \end{minipage}}

        \end{enumerate}
\end{enumerate}

\end{document}
