\documentclass[10pt]{article}

% \usepackage[utf8]{inputenc}
\usepackage{fancyhdr}
\usepackage{amsmath,amssymb,amsthm}
\usepackage{mathrsfs}
\usepackage{enumitem}
\usepackage{verbatim}
\usepackage{parskip}
\usepackage{xcolor}
\usepackage{array} 
\usepackage{url} 
\usepackage{float}
\usepackage{braket}
\usepackage{listings}
\usepackage{inconsolata}
\usepackage{mystyle}
\usepackage{lastpage}
\usepackage{fontspec}
\usepackage[euler-digits,euler-hat-accent]{eulervm}


\raggedbottom
\setmainfont{URW Palladio L}

\begin{document}
\title{\vspace{-2em}Homework 11\vspace{-1em}}
\author{Chris Powell}
\date{}
\maketitle
\thispagestyle{fancy}
\begin{enumerate}[itemsep=1em,label=\Alph*.,leftmargin=*]
    % \setlength{\leftmargini}{.0cm}
    \item Silverman 21.4. Finish the proof of Quadratic Reciprocity (Part II) for the other two cases: primes congruent to $1$ modulo $8$ and primes congruent to $5$ modulo $8$.

                \framebox{\begin{minipage}{\dimexpr\linewidth-2\fboxrule-2\fboxsep}
                    \begin{proof}
                        Suppose $\eq{p}{8}=\eq{1}{8}$. Then $8 \mid p-1$. So for some $k \in \Z$,
                        \[p-1=8k \quad \Rightarrow \quad \frac{p-1}{2} =4k.\]
                        Thus $\#\set{2,4,\ldots,4k}= \#\set{4k+2,4k+4\ldots, 8k}=2k.$
                        Now observe that 
                        \begin{align*}
                            \eq{\tlegendre{2}{p}}{p}&=\eq{ 2^{\frac{p-1}{2}} }{p} \justify{Euler's Criterion}
                            \\&= \eq{ 2}{p}^{\frac{p-1}{2}}  
                            \\&= \eq{-1}{p}^{\frac{p-1}{2}} \justify{"Fundamental formula", page $157$.}
                            \\&= \eq{-1}{p}^{2k}
                            \\&= \eq{(-1)^{2k}}{p}
                            \\&=\eq{1}{p}.
                        \end{align*}
                        Now assume $\eq{p}{8}=\eq{5}{8}$. Then $8 \mid p-5$. So for some $k \in \Z$,
                        \[p-5=8k \quad \Rightarrow \quad p-1=8k+4 \quad \Rightarrow \quad \frac{p-1}{2} =4k+2.\]
                        Thus
                        $\#\set{2,4,\ldots,4k+2}= \#\set{4k+4,4k+4\ldots, 8k+4}=2k+1.$
So
                        \begin{align*}
                            \eq{\tlegendre{2}{p}}{p}&=\eq{ 2^{\frac{p-1}{2}} }{p} \justify{Euler's Criterion}
                            \\&= \eq{ 2}{p}^{\frac{p-1}{2}}  
                            \\&= \eq{-1}{p}^{\frac{p-1}{2}} \justify{"Fundamental formula", page $157$.}
                            \\&= \eq{-1}{p}^{2k+1}
                            \\&= \eq{(-1)^{2k+1}}{p}
                            \\&=\eq{-1}{p}.
                        \end{align*}
                        Hence, $2$ is a QR $\mod{p}$ if $p \equiv 1 \mod{8}$, and $2$ is a NR if $p \equiv 5 \mod{8}$.
                    \end{proof}
                \end{minipage}}


    \pagebreak
    \item Silverman 22.7. Let $p$ be a prime satisfying $p \equiv 3 \mod{4}$ and suppose that $a$ is a quadratic residue modulo $p$.
        \begin{enumerate}[label=(\alph*),leftmargin=*]
            \item Show that $x^{(p+1)/4}$ is a solution to the congruence
                \[x^2 \equiv  \mod{p}.\]
                \framebox{\begin{minipage}{\dimexpr\linewidth-2\fboxrule-2\fboxsep}
            \begin{proof}Observe that
                    \begin{align*}
                        \eq{ \left(a^{\frac{p+1}{4}}\right)^2}{p} &= \eq{a^{\frac{p+1}{2}} }{p}
                        \\&= \eq{a^{\frac{p-1}{2}+1}}{p}
                        \\&= \eq{a^{\frac{p-1}{2}} \cdot a}{p}
                        \\&= \eq{a^{\frac{p-1}{2}}}{p} \cdot \eq{a}{p}
                        \\&= \eq{ \tlegendre{a}{p} }{p} \cdot \eq{a}{p} \justify{Euler's Criterion}
                        \\&= \eq{ 1 }{p} \cdot \eq{a}{p} \justify{$a$ is a QR $\mod{p}$} 
                        \\&= \eq{a}{p}.
                    \end{align*}
                Therefore, $x=a^{\frac{p+1}{4}}$ is a solution to $x^2 \equiv \mod{p}$ for all primes $p \equiv 3 \mod{4}$.    
            \end{proof}
                \end{minipage}}

            \item Find a solution to the congruence $x^2 \equiv 7 \mod{787}$. (Your answer should be between $1$ and $786$.)
                
                \framebox{\begin{minipage}{\dimexpr\linewidth-2\fboxrule-2\fboxsep}
                Since $\eq{787}{4}=\eq{3}{4}$, part (a) of this exercise implies that 
                    \[\eq{7^{\frac{787+1}{4}}}{787} = \eq{7^{144}}{787}\]
                    is a solution. But by successive squaring (i.e., using the $\texttt{expmod(7,144,787)}$ algorithm),
                \[\eq{7^{144}}{787}=\eq{692}{787}.\]
                Thus $x=692$ is solution to $x^2 \equiv 7 \mod{787}$ satisying $1\leq x \leq 786$.
                \end{minipage}}
        \end{enumerate}
   

        \pagebreak
    \item Silverman 22.8. Let $p$ be a prime satisfying $p \equiv 5 \mod{8}$ and suppose that $a$ is a quadratic residue modulo $p$.
        \begin{enumerate}[label=(\alph*),leftmargin=*]
            \item Show that one of the values 
                \[x = a^{(p+3)/8} \quad \textnormal{or} \quad x=2a\cdot(4a)^{(p-5)/8}\]
                is a solution to the congruence
                \[x^2 \equiv a \mod{p}.\]
                \framebox{\begin{minipage}{\dimexpr\linewidth-2\fboxrule-2\fboxsep}
                \begin{proof}
                    Observe that
                    \begin{align*}
                        \eq{\left( a^{\frac{p+3}{8}}\right)^2 }{p} &= \eq{ a^{\frac{p+3}{4}} }{p} = \eq{ a^{\frac{p-1}{4}+1} }{p} = \eq{ a^{\frac{p-1}{4}} \cdot a }{p} = \eq{a}{p}^{\frac{p-1}{4}} \cdot \eq{a}{p}.
                    \end{align*}
                    Since $a$ is a QR $\mod{p}$, we know $\eq{a}{p}=\eq{b^2}{p}$ for some $b \in \Z$, thus
                    \begin{align*}
                        \eq{\left( a^{\frac{p+3}{8}}\right)^2}{p} =  \eq{b^2}{p}^{\frac{p-1}{4}} \cdot \eq{a}{p} = \eq{b^{\frac{p-1}{2}}}{p} \cdot \eq{a}{p}= \eq{\tlegendre{b}{p}}{p} \cdot \eq{a}{p}.
                    \end{align*} 
                    If $b$ is a QR $\mod{p}$, we're done as then 
                    \[\eq{\tlegendre{b}{p}}{p} \cdot \eq{a}{p} = \eq{1}{p} \cdot \eq{a}{p}= \eq{a}{p}.\] 
                    So suppose $b$ is not a QR $\mod{p}$. Then
                    \begin{align*}
                        \eq{\left(2a\cdot(4a)^{\frac{p-5}{8}} \right)^2}{p}
                        &= \eq{4a^2 \cdot (4a)^{\frac{p-5}{4}} }{p}
                        = \eq{4^{\frac{p-5}{4}+1} \cdot a^{\frac{p-5}{4}+1}\cdot a}{p}.  
                    \end{align*}
                    Simplifying further, we get
                    \begin{align*}
\eq{4^{\frac{p-5}{4}+1} \cdot a^{\frac{p-5}{4}+1}\cdot a}{p} = \eq{4^{\frac{p-1}{4}} \cdot a^{\frac{p-1}{4}}\cdot a}{p}
                    =\eq{4}{p}^{\frac{p-1}{4}}\cdot \eq{a}{p}^{\frac{p-1}{4}}\cdot \eq{a}{p}. 
                    \end{align*}
                    But $4$ and $a$ are QR's $\mod{p}$, so we can write that last expression as 
                    \begin{align*}
                     \eq{2^2}{p}^{\frac{p-1}{4}}\cdot \eq{b^2}{p}^{\frac{p-1}{4}}\cdot \eq{a}{p}
                        = \eq{2^{\frac{p-1}{2}}}{p}\cdot \eq{b^{\frac{p-1}{2}}}{p}\cdot \eq{a}{p},
                    \end{align*}
                and by Euler's Criterion, 
                    \begin{align*}
                        \eq{2^{\frac{p-1}{2}}}{p}\cdot \eq{b^{\frac{p-1}{2}}}{p}\cdot \eq{a}{p}    &= \eq{\tlegendre{2}{p}}{p}\cdot \eq{\tlegendre{b}{p}}{p}\cdot \eq{a}{p}
                    \end{align*}
                        Finally, as $\eq{p}{8}=\eq{5}{8}$ and $b$ is a QR $\mod{p}$, the last expression gives
                        \begin{align*}
                            \eq{-1}{p}\cdot \eq{-1}{p}\cdot \eq{a}{p}=\eq{(-1) \cdot (-1) \cdot a}{p}=\eq{ 1 \cdot a}{p}= \eq{a}{p}.
                    \end{align*}
                \end{proof}
                \end{minipage}} 
           
            \pagebreak
            \item Find a solution to the congruence $x^2 \equiv 5 \mod{541}$. (Give an answer lying between $1$ and $540$.) 
           

                \framebox{\begin{minipage}{\dimexpr\linewidth-2\fboxrule-2\fboxsep}
                First, note that the modulus $541$ is a prime satisfying $\eq{541}{8}=\eq{5}{8}$, so we can apply the above result. Since
                \begin{align*}
                    \eq{5^{\frac{541-1}{4}}}{541} &=  \eq{5^{\frac{540}{4}}}{541} = \eq{5^{135}}{541} = \eq{1}{541},
                \end{align*}
                    we know $x=5^{\frac{541+3}{8}}$ is a solution. But
                    \begin{align*}
                        \eq{5^{\frac{541+3}{8}}}{541} =\eq{5^{\frac{544}{8}}}{541}=\eq{5^{68}}{541},
                    \end{align*}
                    and by successive squaring, we obtain
                    \[\eq{5^{68}}{541}=\eq{345}{541}.\]
                    Hence, $x=345$ is solution to $x^2 \equiv 5 \mod{541}$ satisfying $1 \leq x \leq 540$.
                \end{minipage}}

            \item Find a solution to the congruence $x^2 \equiv 13 \mod{653}$. (Give an answer lying between $1$ and $652$.)


    \framebox{\begin{minipage}{\dimexpr\linewidth-2\fboxrule-2\fboxsep}
                First, note that the modulus $653$ is a prime satisfying $\eq{543}{8}=\eq{5}{8}$, so we can again apply the above result. As 
                \begin{align*}
                    \eq{13^{\frac{653-1}{4}}}{643} &=  \eq{13^{\frac{652}{4}}}{653} = \eq{13^{163}}{653} = \eq{-1}{653},
                \end{align*}
                    we know $x=(2\cdot 13)\cdot (4 \cdot 13)^{\frac{653-5}{8}}$ is a solution. But
                    \begin{align*}
                        \eq{(2\cdot 13)\cdot (4 \cdot 13)^{\frac{653-5}{8}}}{653}= \eq{(26)\cdot (54)^{\frac{648}{8}}}{653} = \eq{(26)\cdot (54)^{81}}{653},
                    \end{align*}
                    and
                    \begin{align*}
                        \eq{26\cdot52^{81}}{653}&=\eq{26}{653}\cdot\eq{52^{81}}{653} 
                        \\&=\eq{26}{653}\cdot \eq{212}{653} \justify{successive squaring}
                        \\&=\eq{26\cdot 212}{653}
                        \\&=\eq{5512}{653}
                        \\&=\eq{288}{653}.
                    \end{align*}
                    Hence, $x=288$ is solution to $x^2 \equiv 13 \mod{653}$ satisfying $1 \leq x \leq 652$.
                \end{minipage}}


\end{enumerate}
 k
  \pagebreak
    \item Silverman 22.9. 
            Let $p$ be a prime that is congruent to $5$ modulo $8$. Write a program to solve the congruence
            \[x^2 \equiv a \mod{p}\]
            using the method described in the previous exercise and successive squaring. The output should be a solution satisfying $0 \leq x <p$. Be sure to check that $a$ is a quadratic residue, and return an error message if it is not. Use your program to solve the congruences
            \[x^2 \equiv 17 \mod{1021}, \quad x^2 \equiv 23 \mod{1021}, \quad x^2 \equiv 31 \mod{1021}.\]

\begin{lstlisting}
def expmod(a, k, m):
    """compute a^k mod m"""
    b = 1
    while k:
        if k % 2 == 1:
            b = (b * a) % m
        a, k = (a ** 2) % m, k // 2
    return b

def residue(n, p):
    """Return each a ≡ b^n mod p for some integer b and modulus p; n=2,3"""
    R = []
    for i in range(p):
        R += [(i**n) % p]
    return set(R)


def hw11(a, p):
    """Return solution-pair to x^2=a (mod p) for prime p=5 (mod 8)"""
    if a not in residue(2, p):
        return str(a)+" is not a quadratic residue modulo "+str(p)+"." 
    if expmod(a, ((p - 1) // 4), p) == 1:
        x = expmod(a, ((p + 3) // 8), p)
        return x, -x % p
    else:
        x = ((2 * a) * expmod(4 * a, ((p - 5) // 8), p)) % p
        return x, -x % p
\end{lstlisting}

\begin{proof}
    Since the \texttt{residue()} and \texttt{expmod()} algorithms have already been shown to terminate and return the correct output, it immediately follows that that \texttt{hw11()} algorithm must terminate. It remains to show correctness. Let $a$ and $p$ be the respective values of the python variables $\texttt{a}$ and $\texttt{p}$. Assume $p$ is prime such that $p \equiv 5 \mod{8}$. Now either $a$ is a quadratic residue modulo $p$, or it isn't. If $a$ is not, then by correctness of \texttt{residue()}, the first \texttt{if} statement will evaluate to true and the program will return an error message; otherwise, it will evaluate to false and execution proceeds. So suppose $a$ is a quadratic residue modulo $p$. Then by Exercise C, every solution to the congruence $x^2 \equiv 5 \mod{p}$ is of the form
    \[a^{\frac{p+3}{8}} \quad \textnormal{or} \quad 2a\cdot(4a)^{(p-5)/8},\] depending on whether 
    \[\eq{a^{\frac{p-1}{4}}}{p} = \eq{1}{p} \quad \textnormal{or} \quad \eq{a^{\frac{p-1}{4}}}{p} = \eq{-1}{p}.\]
    But those are exactly the two remaining termination conditions, and in each case the program returns the appropriate form by correctness of \texttt{expmod()}. Hence the program returns correctly.
\end{proof}


    \framebox{\begin{minipage}{\dimexpr\linewidth-2\fboxrule-2\fboxsep}
        $\texttt{hw11(17,1021)}$ returns $\texttt{(494,527)}$ 
        \\$\texttt{hw11(23,1021)}$ returns $\texttt{(858,163)}$
        \\$\texttt{hw11(31,1021)}$ returns \texttt{
'31 is not a quadratic residue modulo 1021.'
} 
    \end{minipage}}

\end{enumerate}
\end{document}
