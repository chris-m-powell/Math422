\documentclass[10pt,twoside]{article}

% \usepackage[utf8]{inputenc}
\usepackage{geometry}
\usepackage{fancyhdr}
\usepackage{amsmath,amssymb,amsthm}
\usepackage{mathrsfs}
\usepackage{enumitem}
\usepackage{verbatim}
\usepackage{parskip}
\usepackage{xcolor}
\usepackage{array} 
\usepackage{url} 
\usepackage{float}
\usepackage{braket}
\usepackage{listings}
\usepackage{inconsolata}
\usepackage{mystyle}
\usepackage{lastpage}
\usepackage{fontspec}
\usepackage[euler-digits,euler-hat-accent]{eulervm}


\raggedbottom
\setmainfont{URW Palladio L}
% \newfontfamily\headingfont[Path = /usr/share/fonts/opentype/bebas-neue/]{BebasNeue Regular.otf}
% \setmainfont[Path = /usr/share/fonts/opentype/tex-gyre-pagella/]{texgyrepagella-regular.otf}
\begin{document}
\title{\vspace{-2em}Homework 9\vspace{-1em}}
\author{Chris Powell}
\date{}
\maketitle
\thispagestyle{fancy}

\begin{enumerate}[itemsep=2em,label=\Alph*.]
    \item Silverman 17.5.
        \begin{enumerate}[label=(\alph*)]
            \item Try to use the methods in this chapter to compute the square root of $23$ modulo $1279$. (The number $1279$ is prime.) What goes wrong?
    
                        \framebox{\begin{minipage}{\dimexpr\linewidth-2\fboxrule-2\fboxsep}
            Consider the congruence $x^2 \equiv 23 \mod{1279}$. As $1279$ is prime,
                    \[\varphi(1279)=1279^1-1279^0=1279-1 = 1278.\]
                But by applying the Euclidean algorithm, we find that 
                \[\gcd(2,\varphi(1279))=\gcd(2,1728)=2 >1.\]
                So by Exercise 17.3.b, either $x^2 \equiv 23 \mod{1279}$ has no solution or it has at least two solutions.
                        \end{minipage}}

            \item More generally, if $p$ is an odd prime, explain why the methods in this chapter cannot be used to find square roots modulo $p$. We will investigate the problem of square roots modulo $p$ in later chapters.
           
                        \framebox{\begin{minipage}{\dimexpr\linewidth-2\fboxrule-2\fboxsep}
Consider the congruence $x^2 \equiv b \mod{p}$. Assume $p$ is an odd prime. Then
                \[\varphi(p)=p^1-p^0=p-1 \equiv 0 \mod{2}.\]
                So $\gcd(2,\varphi(p))=2>1$. Thus by Exercise 17.3.b, either$x^2 \equiv b \mod{p}$ has no solution or it has at least two solutions.
                        \end{minipage}}

            \item Even more generally, explain why our method for computing $k$\textsuperscript{th} roots modulo $m$ does not work if $\gcd(k,\varphi(m))>1$.

                        \framebox{\begin{minipage}{\dimexpr\linewidth-2\fboxrule-2\fboxsep}
                            Consider the congruence $x^k \equiv b \mod{m}$. Suppose $g=\gcd(2,\varphi(m))>1$. Let $u,v \in \Z$. Then since $g \mid k, \varphi(m)$, $g \mid ku-\varphi(m)v$. Thus $ku-\varphi(m)v = g \cdot d$ for some $d \in \Z$. But $g >1$, so $g \nmid ku-\varphi(m)v$.  
                        \end{minipage}}


        \end{enumerate}
\pagebreak
    \item Silverman 17.6. Write a program to solve $x^k \equiv b \mod{m}$.
    
        \begin{lstlisting}
def gcd(a, b):
    """Return gcd(a,b)"""
    while b:
        a, b = b, a % b
    return a


def xgcd(a, b):
    """Return (g, x, y) such that a*x + b*y = g = gcd(a, b)"""
    if b == 0:
        return a, 1, 0
    x, g, v, w = 1, a, 0, b
    while w != 0:
        x, g, v, w = v, w, x - (g // w) * v, g % w
    x = x % (b // g)
    return g, x, (g - (a * x)) // b

        
def expmod(a, k, m):
    """compute a^k mod m"""
    b = 1
    while k:
        if k % 2 == 1:
            b = (b * a) % m
        a, k = (a ** 2) % m, k // 2
    return b

        
def modroot(k, b, m):
    """return x such that x^k cong b mod m if gcd(b,m)=1, gcd(k, totient(m)"""
    if gcd(b, m) == 1:
        (g, u, v) = xgcd(k, totient(m))
        print(g, u, v)
        if g == 1:
            return expmod(b, u, m)
        \end{lstlisting}
                \framebox{\begin{minipage}{\dimexpr\linewidth-2\fboxrule-2\fboxsep}
   Note that \texttt{modroot} requires $\gcd(b,m)=\gcd(k,\varphi(m))=1$.
                \end{minipage}}

\pagebreak
    \item Silverman 18.2. It may appear that RSA decryption does not work if you are unlucky enough to choose a message $a$ that is not relatively prime to $m$. Of course, if $m =pq$ and $p$ and $q$ are large, this is very unlikely to occur.
        \begin{enumerate}[label=(\alph*)]
            \item Show that in fact RSA decryption does work for all messages $a$, regardless of whether or not they have a factor in common with $m$. 
    
                \framebox{\begin{minipage}{\dimexpr\linewidth-2\fboxrule-2\fboxsep}
                    If $m=pq$, where $p$ and $q$ are distinct primes, then provided $\gcd(k,\varphi(m))=1$, Exercise 17.4.a implies that $x \equiv b^u \mod{m}$ is always a solution to $x^k \equiv b \mod{m}$ (even if $\gcd(b,m)> 1$). Thus RSA decryption works for all messages $a$, regardless of whether or not they have a common factor.    
                \end{minipage}}
               
        
            \item More generally, show that RSA decryption works for all messages $a$ as long as $m$ is a product of distinct primes.
            
                \framebox{\begin{minipage}{\dimexpr\linewidth-2\fboxrule-2\fboxsep}
                    By the Fundantemal Theorem of Arithmetic, we know the RSA modulus $m$ can be factored into a product of distinct primes. Assume $k$ is such that $\gcd(k,\varphi(m))=1$. Then this statement follows from the Exercise 17.4, which we proved in the previous homework assignment.    
                \end{minipage}}
            
            \item Give an example with $m=18$ and $a=3$ where RSA decryption does work. [Remember, $k$ must be chosen relatively prime to $\varphi(m)=6$.]

                \framebox{\begin{minipage}{\dimexpr\linewidth-2\fboxrule-2\fboxsep}
                Observe that
                    \[\varphi(m)=\varphi(18)=\varphi(2)\varphi(3^2)= (2-1)(3^2-3^1)= 6.\]
                Note that $k=5$ is the least positive integer for which $\gcd(k,6)=1$. So
                \[a^5 = 3^5 \equiv 9 \mod{18}.\]
                By applying the extended Euclidean algorithm we find that $(u,v)=(5,4)$ satisfies
                \[5u-6v=1.\]
                \end{minipage}}
                \end{enumerate}
\pagebreak
            \item Silverman 18.4. Here are two longer messages to decode if you like to use computers.
        \begin{enumerate}[label=(\alph*)]
            \item You have been sent the following message:
                
                \ttfamily
                \begin{center}
                \begin{tabular}{ c c c }
                    5272281348,      & 21089283929,   & 311723025,     \\
                    26945939925,     & 26844144908,   & 22890519533,   \\
                    27395704341,     & 2253724391,    & 1481682985,    \\
                    2163791130,      & 13583590307,   & 5838404872,    \\
                    12165330281,     & 28372578777,   & 7536755222,
\end{tabular}
                \end{center}
               \normalsize\rmfamily

                It has been encoded using $p=187963$, $q=163841$, $m=pq=30796045883$, and $k=48611$. Decode the message.
           
            
                \begin{lstlisting}
def modrootpq(k, b, m, p, q):
    """given p,q, return x such that x^k cong b mod m=pq if gcd(b,m)=1, gcd(k, totient(m))"""
    if gcd(b, m) == 1:
        (g, u, v) = xgcd(k, totient(p)*totient(q))
        if g == 1:
            return expmod(b, u, m)


def rsa_decrypt(k, B, m, p, q):
    "Decrypt RSA ciphertext B given k, m, p, q"
    a = ""
    for b in B:
        a += str(modrootpq(k, b, m, p, q))
    A, a = [chr(int(a)+54) for a in [a[i:i + 2] for i in range(0, len(a), 2)]], ""
    for i in A:
        a += i
    return a
    

B = [5272281348, 21089283929, 3117723025, 26844144908, 22890519533,
    26945939925, 27395704341, 2253724391, 1481682985, 2163791130,
    13583590307, 5838404872, 12165330281, 28372578777, 7536755222]
k = 48611
m = 30796045883
p = 187963
q = 163841

print(rsa_decrypt(k, B, m, p, q))
                \end{lstlisting}

                \framebox{\begin{minipage}{\dimexpr\linewidth-2\fboxrule-2\fboxsep}
                \texttt{MATHEMATICSISTHEQUEENOFSCIENCEANDNUMBERTHEORY
                    \\ISTHEQUEENOFMATHEMATICSKFGAUSS}            
                \end{minipage}} 
            \item You intercept the following message, which you know has been encoded using the modulus 
                \[m=956331992007843552652604425031376690367\] 
                and exponent $k=12398737$. Break the code and decipher the message.

                \ttfamily
                \begin{center}
                \begin{tabular}{c}
                    821566670681253393182493050080875560504, 
\\87074173129046399720949786958511391052, 
\\552100909946781566365272088688468880029, 
\\491078995197839451033115784866534122828, 
                    \\172219665767314444215921020847762293421.
                \end{tabular}
                \end{center}
        \rmfamily
        \end{enumerate}
    
    \pagebreak
    \item Silverman 12.1. Start with the list consisting of a single prime $\set{5}$ and use ideas in Euclid's proof that there infinitely many primes to create a list of primes until the numbers get too large for you to easily factor. (You should be able to factor any number less than $1000$.) 
    
    
                        \framebox{\begin{minipage}{\dimexpr\linewidth-2\fboxrule-2\fboxsep}
                            Let $S_0 =\set{5}$. We use Euclid's idea to proceed as follows:
                            \begin{align*}
                                A_1&=5+1=6=2 \cdot 3                                & S_1&=S_0\cup\set{2,3}
                                \\A_2&=(2 \cdot 3 \cdot 5)+1=31                     & S_2&=S_1\cup\set{31}
                                \\A_3&=(2 \cdot 3 \cdot 5 \cdot 31) +1 =931=7^2 \cdot 19 & S_3&=S_2 \cup\set{7,19}
                                \\A_4&=(2 \cdot 3 \cdot 5 \cdot 7 \cdot 19 \cdot 31 ) +1 =123691
                            \end{align*}
                            Since $123691 >1000$, and thus difficult to factor, the list of primes is 
                            \[\bigcup_{i=0}^3 S_i = \set{2,3,5,7,19,31}.\]
                        \end{minipage}} 
    \pagebreak
    \item Silverman 12.2.
        \begin{enumerate}[label=(\alph*)]
            \item Show that there are infinitely many primes that are congruent to $5$ modulo $6$. [\textit{Hint}. Use $A=6p_1p_2\cdots p_r + 5)$.]
         \vspace{1em}

                        \framebox{\begin{minipage}{\dimexpr\linewidth-2\fboxrule-2\fboxsep}
                \begin{lemma*}Let $p>3$ be prime integer. Then $[p]_6 \in \set{[1]_6,[5]_6}$.
                \end{lemma*}
                            \begin{proof}Since $\Z_6= \set{[0]_6,\ldots,[5]_6}$ forms a partition of $\Z$, we know $[p]_6=[x]_6$ for exactly one $x \in \Z_6$.
                    Suppose $[p]_6 = [0]_6$. Then $p=6k=2(3k)$ for some $k \in \Z$. So $p>3$ is even, contradicting primality of $p$. The argument is similiar for $[p]_6 \in \set{[2]_6,[3]_6,[4]_6}$. Therefore, $[p]_6\in \set{[1]_6,[5]_6}$.
                \end{proof}

                \begin{proof}Suppose there are only finitely many primes congruent to $5$ modulo $6$. Let 
                    \[S=\set{5,p_1\ldots,p_r}\] 
                    be those primes. Consider $A=6\left(\prod_{i=1}^r p_i\right)+5$. We know $A$ is not prime since $[A]_6 = [5]_6$, but $A \notin S$. So $A$ has prime factorization $A=\prod_{i=1}^s q_i$, where each $q_i$ is a distinct. I claim that $[q_i]_6 = [5]_6$ for some $i$. Suppose $q_i \mid 6$ for some $i$. Then since $q_i \mid A$, 
                    \[q_i \mid A - 6 \left( \prod_{i=1}^r p_i \right)=5.\]
                    Thus $q_i = 5$ since the only prime divisor of $5$ is $5$. But this is a contradiciton since $5 \nmid 6$. Thus $2,3 \notin S$ and so $[q_i]_6 \in \set{[1]_6,[5]_6}$ for all $i$, by the above lemma. Now it cannot be that $[q_i]_6=[1]_6$ for all $i$, as then $[A]_6=[1]_6$, a contradiction. So there must exist some $q_k$ for which $[q_k]_6=[5]_6$, as claimed. But clearly $q_k \mid A$, and yet no $x \in S$ divides $A$ by construction. So $q_k \notin S$. Therefore, our original suppositon that there are finitely many primes congruent to $5$ modulo $6$ is false. The result follows.
                \end{proof}
                        \end{minipage}} 
            
            \item Try to use the same idea (with $A=5p_1p_2 \cdots p_r +4)$ to show that there are infinitely many primes congruent to $4$ modulo $5$. What goes wrong? In particular, what happens if you start with $\set{19}$ and try to make a longer list?

                        \framebox{\begin{minipage}{\dimexpr\linewidth-2\fboxrule-2\fboxsep}
                In the proof of part (a), we showed that if $A=m\left(\prod_{i=1}^r p_i \right)+b$ is not prime, then at least one of its prime factors is congruent to $b$ modulo $m$, where $(b,m)=(5,6)$. However, this is not the case for $(b,m)=(4,5)$. Observe that if we start with $\set{19}$ and compute 
                \[A=4(19)+5=99=3^2 \cdot 11^1,\]
                            we see that $A$ is not prime and none of its prime factors are congruent to $4 \mod{5}$: $[11]_5=[1]_5$. Note that this prime factorization of $A$ is unique (up to rearrangement). So the argument used in part (a) cannot be applied to this case.
                        \end{minipage}} 
        \end{enumerate}
\end{enumerate}

\end{document}
